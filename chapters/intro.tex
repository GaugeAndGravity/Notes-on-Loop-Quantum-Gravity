% !TeX root = ../NotesOnLQG.tex

\chapter{简介}

	在现代物理学中,我们已经认识到自然界存在四种相互作用,即电磁力、强相互作用、弱相互作用和引力,其中前三种都可以纳入著名的标准模型,使用规范场论的语言描述其量子理论,而引力却纳入标准模型的框架。基于量子场论和标准模型的巨大成功,物理学家普遍相信自然界在基本理论层面应当是量子化的,然而当前最好的引力理论仍然是描述经典引力的广义相对论,它描述了时空与经典物质场的相互作用,其中时空是4维光滑的洛伦兹流形,而物质场和物理量则是时空上各种光滑的场,时空上的度规和物质场通过爱因斯坦方程耦合。这样的物理图像与量子场论是不相容的,即便是弯曲时空量子场论,也是将度规 $\tensor{g}{_a_b}$ 视为背景几何而不是动力学量,并且场论的基本代数的定义常常要依赖一个作为背景的 $\tensor{g}{_a_b}$。另一方面,试图将引力理论本身按照通常微扰量子场论方式量子化,即考虑将度规分解为背景和动力学部分 $\tensor{g}{_a_b} = \tensor{\eta}{_a_b} + \tensor{h}{_a_b}$,并以 $\tensor{h}{_a_b}$ 为动力学场建立微扰量子场论的方法会遇到不可重整等问题。于是,能否建立一个自洽的、背景独立、非微扰的量子引力理论,是现代物理学中一个非常值得研究的问题。

	除了出于统一的考虑之外,现代物理学中还存在如下一些问题,这些问题很可能会被量子引力理论所自然解决:
	\begin{itemize}
		\item \emph{经典-量子不相容}
		
			如果我们研究类似早期宇宙这样的问题,几何无法被视为背景,必须考虑爱因斯坦方程
			\begin{equation}
				\tensor{R}{_a_b} - \frac{1}{2} R \tensor{g}{_a_b} = \gkappa \tensor{T}{_a_b},
			\end{equation}
			右侧的物质场能动量又必须用量子场论给出,于是考虑先指定一个固定背景 $g_0$,计算出 $\left\langle \tensor{\hat{T}}{_a_b}(g_0) \right\rangle$,代入爱因斯坦方程后求得 $g_1$,再作为背景重复迭代。然而,这样做是不收敛的\cite{Flanagan:1996gw}。解决此问题需要构造背景独立的量子场论,并把引力理论量子化,然后考察“量子爱因斯坦方程”。

		\item \emph{广义相对论的奇异性}
		
			众所周知,霍金和彭罗斯证明了一组奇异性定理,即广义相对论中只要满足了物理上十分合理的一些条件,如各种能量条件,则广义相对论预言必然存在奇异性\cite{Hawking1973,wald1989}。奇异性被相信是广义相对论不完善的证据,例如在黑洞内或大爆炸的奇点处,物理定律完全失效。很多物理学家认为量子引力将可以解决此问题。

		\item \emph{量子场论的发散问题}
		
			在量子场论中存在紫外发散,即大动量/小尺度引起的发散。很多物理学家期望在普朗克尺度时空具有离散的结构,从而提供一个自然的截断,这需要再次革新时空观,一个量子引力理论有可能提供这种自然截断。

		\item \emph{黑洞熵}
		
			通过量子引力的第一性原理计算给出黑洞的贝肯斯坦-霍金熵被认为是量子引力理论的检验标准之一,弦论已经在此问题上有了很大的突破。
	\end{itemize}

	本文所要介绍的是当前量子引力理论候选者之一的圈量子引力(loop quantum gravity,常缩写为LQG),它得名于量子化过程受规范场论的 Wilson loops 启发。该理论也常被 Thomas Thiemann 称为量子广义相对论\cite{Thiemann2007},因为它保留了广义相对论的微分同胚不变性和背景独立性两大基本原理。目前它已经像熟知的量子场论那样发展出了正则量子化和协变量子化(路径积分量子化)两套体系,一般圈量子引力指正则圈量子引力,协变圈量子引力常称为 spinfoam 模型。

	1986-1987 年间问世的 Ashtekar 变量改写了广义相对论,它是圈量子引力理论的基石,自Ashtekar 变量问世后,1988-1990 年间,Carlo Rovelli 和 Lee Smolin 即获得了圈量子引力的 Hilbert 空间的一组基底\cite{Rovelli1988,Rovelli1989},称为 spin network state,将在后文提及。1994 年,Rovelli 和 Smolin 进一步构造了理论中的几何量算符,并得到了离散谱\cite{Rovelli1994},因此圈量子引力可以为场论提供自然截断。 1998年 Ashtekar 与 Kirill Krasnov 建立了黑洞自由度与其边界上的 Chern-Simons 场之间的联系,对黑洞熵给出了与贝肯斯坦-霍金熵只差一个自由选择的比例系数的结果\cite{Ashtekar:1998ue},一个更新的计算可参见\cite{Thiemann2007} 的第15章。近年来,圈量子引力中关于黑洞奇异性有很多讨论\cite{Ashtekar:2018lag,Ashtekar:2018cay,Ashtekar:2005qt,Bohmer:2007wi,Olmedo:2017lvt},结果表明黑洞内部的奇异性在圈量子引力中得到避免。将圈量子引力的方法用于宇宙学,即圈量子宇宙学(loop quantum cosmology)也得到了同样的结论,“大反弹”替代了大爆炸\cite{Ashtekar:2011ni}。

	近年被广泛研究的 Spinfoam 模型的根源是广义相对论作为具有约束的拓扑场论的表述\cite{Bianchi2017}。这是一种具有有限自由度的理论,并且其自由度与4流形的三角剖分中围绕二维缺陷的Lorentz联络的Wilson loop相关。该理论提供了正则圈量子引力的协变形式,可以视为一种路径积分表述。
	
	本文第二章中介绍了引力理论的正则表述,包括最早的 ADM formulation 及对其量子化的尝试(量子几何动力学),以及在此之后的规范理论表述——Palatini作用量、Holst作用量及1986-1987 年间问世的 Ashtekar 变量。第三章介绍量子化的程序,之后第四、五章按照此量子化程序定义正则圈量子引力。第六章简要介绍量子几何算符和相干态,这同时也是为之后介绍 spinfoam 模型做准备。第七章介绍 spinfoam 模型,或称协变圈量子引力理论,主要围绕跃迁振幅的导出展开。
