% !TeX root = ../NotesOnLQG.tex

\chapter{第\ref{chp-canonical_gravity}章中的计算}

	\section{ADM formulation}

		\begin{Property}
			\label{pro_EEq}
			易证明,Einstein-Hilbert 作用量
			\begin{equation}
				S_{\text{EH}}[g] = \frac{1}{2\gkappa} \int_M \curR[g] 
			\end{equation}
			的运动方程为真空 Einstein 方程
			\begin{equation}
				Ric - \frac{1}{2} \curR g = 0,
			\end{equation}
			或采取抽象指标形式,写作
			\begin{equation}
				\tensor{\Ric}{_a_b} - \frac{1}{2} \curR \tensor{g}{_a_b} = 0.
			\end{equation}
		\end{Property}

		\begin{Proof}
			\label{prf_EEq}
			考虑时空 $(M,\tensor{g}{_a_b})$ 及 $M$ 上的一族度规 $\tensor{g}{_a_b}(\lambda)$,满足 $\tensor{g}{_a_b}(0)=\tensor{g}{_a_b}$,则任何依赖度规的量 $T$ 的变分为
			\begin{equation}
				\var T\left( \tensor{g}{_a_b} \right) \definedby \left. \dv{T\left( \tensor{g}{_a_b}(\lambda) \right)}{\lambda}  \right|_{\lambda=0},
			\end{equation}
			对 $\tensor{\delta}{^a_b} = \tensor{g}{^a^c} \tensor{g}{_c_b}$ 两边变分得关系
			\begin{equation}
				\begin{split}
					\var{\tensor{g}{_a_b}} = - \tensor{g}{_a_c} \tensor{g}{_b_d} \var{\tensor{g}{^c^d}} \qc \var{\tensor{g}{^a^b}} = - \tensor{g}{^a^c} \tensor{g}{^b^d} \var{\tensor{g}{_c_d}}.
				\end{split}
			\end{equation}

			注意到
			\begin{equation}
				\var{\Lad} = \underbrace{\sqrt{-g} \tensor{g}{^a^b} \var{\tensor{R}{_a_b}}}_{\RomanNumeralCaps{1}} + \underbrace{\sqrt{-g} \tensor{R}{_a_b} \var{\tensor{g}{^a^b}}}_{\RomanNumeralCaps{2}}
			\end{equation}
			考虑与 $\tensor{g}{_a_b}(\lambda)$ 适配的导数算符 $\tensor{\myvaried{\nabla}}{_a}$,假定与 $\Nabla{a}$ 相差 $\tensor{C}{^c_a_b}(\lambda)$,即
			\begin{equation}
				\left( \Nabla{a} - \tensor{\myvaried{\nabla}}{_a} \right) \tensor{\omega}{_b} = \tensor{C}{^c_a_b}(\lambda) \tensor{\omega}{_c},
			\end{equation}
			通过 $\tensor{\myvaried{\nabla}}{_a} \tensor{g}{_b_c}(\lambda) = 0$ ,与计算克氏符类似,可算得
			\begin{equation}
				\tensor{C}{^c_a_b}(\lambda) = \frac{1}{2} \tensor{g}{^c^d}(\lambda) \left( \Nabla{a} \tensor{g}{_b_d}(\lambda) + \Nabla{b} \tensor{g}{_a_d}(\lambda) - \Nabla{d} \tensor{g}{_a_b}(\lambda) \right),\label{eq-Clambda}
			\end{equation}
			进而考虑 $\tensor{\myvaried{\nabla}}{_a}$ 相应的黎曼张量 $\tensor{R}{_a_b_c^d}(\lambda)$,按照定义,有
			\begin{equation}
				\begin{split}
					\tensor{R}{_a_b_c^d}(\lambda) \tensor{\omega}{_d} ={}& 2 \tensor{\myvaried{\nabla}}{_{[a}} \tensor{\myvaried{\nabla}}{_{b]}} \tensor{\omega}{_c}\\
					={}& \tensor{\myvaried{\nabla}}{_a} \left( \Nabla{b} \tensor{\omega}{_c} - \tensor{C}{^d_b_c}(\lambda) \tensor{\omega}{_d} \right) - \tensor{\myvaried{\nabla}}{_b} \left( \Nabla{a} \tensor{\omega}{_c} - \tensor{C}{^d_a_c}(\lambda) \tensor{\omega}{_d} \right)\\
					={}& \left( {\color{red}\Nabla{a} \Nabla{b}\tensor{\omega}{_c}} - {\color{blue}\tensor{C}{^d_a_b}(\lambda) \Nabla{d} \tensor{\omega}{_c}} - \tensor{C}{^d_a_c}(\lambda) \Nabla{b} \tensor{\omega}{_d} \right)\\
					& {}- \left[ \Nabla{a} \left( \tensor{C}{^d_b_c}(\lambda) \tensor{\omega}{_d} \right) - {\color{green}\tensor{C}{^e_a_b}(\lambda) \tensor{C}{^d_e_c}(\lambda) \tensor{\omega}{_d}} - \tensor{C}{^e_a_c}(\lambda) \tensor{C}{^d_b_e}(\lambda) \tensor{\omega}{_d} \right]\\
					& {}- \left( {\color{red}\Nabla{b} \Nabla{a}\tensor{\omega}{_c}} - {\color{blue}\tensor{C}{^d_b_a}(\lambda) \Nabla{d} \tensor{\omega}{_c}} - \tensor{C}{^d_b_c}(\lambda) \Nabla{a} \tensor{\omega}{_d} \right)\\
					& {} + \left[ \Nabla{b} \left( \tensor{C}{^d_a_c}(\lambda) \tensor{\omega}{_d} \right) - {\color{green}\tensor{C}{^e_b_a}(\lambda) \tensor{C}{^d_e_c}(\lambda) \tensor{\omega}{_d}} - \tensor{C}{^e_b_c}(\lambda) \tensor{C}{^d_a_e}(\lambda) \tensor{\omega}{_d} \right]\\
					={} & {\color{red} \tensor{R}{_a_b_c^d} \tensor{\omega}{_d}} - 2 \left( \Nabla{{[a|}} \tensor{C}{^d_{b]}_c}(\lambda) \right) \tensor{\omega}{_d} + 2 \tensor{C}{^e_{[a}_{|c|}}(\lambda) \tensor{C}{^d_{b]}_e}(\lambda) \tensor{\omega}{_d},
				\end{split}
			\end{equation}
			即
			\begin{gather}
				\tensor{R}{_a_b_c^d}(\lambda) = \tensor{R}{_a_b_c^d} - 2 \Nabla{{[a}} \tensor{C}{^d_{b]}_c}(\lambda) + 2 \tensor{C}{^e_c_{[a}}(\lambda) \tensor{C}{^d_{b]}_e}(\lambda),\\
				\tensor{R}{_a_b}(\lambda) = \tensor{R}{_a_c_b^c}(\lambda) = \tensor{R}{_a_b} - 2 \Nabla{{[a}} \tensor{C}{^c_{c]}_b}(\lambda) + 2 \tensor{C}{^e_b_{[a}}(\lambda) \tensor{C}{^c_{c]}_e}(\lambda),
			\end{gather}
			由于 $\tensor{C}{^c_a_b}(0)=0$,求导得
			\begin{equation}
				\var{\tensor{R}{_a_b}} = - 2 \Nabla{{[a}} \var \tensor{C}{^c_{c]}_b} = \Nabla{c} \var \tensor{C}{^c_a_b} - \Nabla{a} \var \tensor{C}{^c_c_b},\label{eq-varR_varC}
			\end{equation}
			而由~\eqref{eq-Clambda}得
			\begin{equation}
				\begin{split}
					\var{\tensor{C}{^c_a_b}} &= \frac{1}{2} \var\tensor{g}{^c^d} \left( \Nabla{a} \tensor{g}{_b_d} + \Nabla{b} \tensor{g}{_a_d} - \Nabla{d} \tensor{g}{_a_b} \right) + \frac{1}{2} \tensor{g}{^c^d} \left( \Nabla{a} \var \tensor{g}{_b_d} + \Nabla{b} \var \tensor{g}{_a_d} - \Nabla{d} \var \tensor{g}{_a_b} \right)\\
					&= \frac{1}{2} \tensor{g}{^c^d} \left( \Nabla{a} \var \tensor{g}{_b_d} + \Nabla{b} \var \tensor{g}{_a_d} - \Nabla{d} \var \tensor{g}{_a_b} \right),\\
					\var{\tensor{C}{^c_c_b}} &= \frac{1}{2} \tensor{g}{^c^d} \left( \Nabla{c} \var \tensor{g}{_b_d} + \Nabla{b} \var \tensor{g}{_c_d} - \Nabla{d} \var \tensor{g}{_c_b} \right)\\
					&= \frac{1}{2} \tensor{g}{^c^d} \Nabla{b} \var \tensor{g}{_c_d},
				\end{split}
			\end{equation}
			代入~\eqref{eq-varR_varC} 得
			\begin{equation}
				\begin{split}
					\var{\tensor{R}{_a_b}} &= \frac{1}{2} \tensor{g}{^c^d} \Nabla{c} \left( \Nabla{a} \var \tensor{g}{_b_d} + \Nabla{b} \var \tensor{g}{_a_d} - \Nabla{d} \var \tensor{g}{_a_b} \right) - \frac{1}{2} \tensor{g}{^c^d} \Nabla{a} \Nabla{b} \var \tensor{g}{_c_d}\\
					&= \frac{1}{2} \tensor{g}{^c^d} \left( \Nabla{c} \Nabla{a} \var \tensor{g}{_b_d} + \Nabla{c} \Nabla{b} \var \tensor{g}{_a_d} - \Nabla{c} \Nabla{d} \var \tensor{g}{_a_b} - \Nabla{a} \Nabla{b} \var \tensor{g}{_c_d} \right),
				\end{split}
			\end{equation}
			则
			\begin{equation}
				\begin{split}
					\var{R} = 
				\end{split}
			\end{equation}
		\end{Proof}
