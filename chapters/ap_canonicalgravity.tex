% !TeX root = ../NotesOnLQG.tex

\chapter{第\ref{chp-canonical_gravity}章中的计算}

	\section{ADM formulation}

		\begin{Property}[\pageref{eq-EEq}页\eqref{eq-EEq}]
			\label{pro_EEq}
			易证明,Einstein-Hilbert 作用量
			\begin{equation}
				S_{\text{EH}}[g] = \frac{1}{2\gkappa} \int_M \curR[g] 
			\end{equation}
			的运动方程为真空 Einstein 方程
			\begin{equation}
				Ric - \frac{1}{2} \curR g = 0,
			\end{equation}
			或采取抽象指标形式,写作
			\begin{equation}
				\tensor{\Ric}{_a_b} - \frac{1}{2} \curR \tensor{g}{_a_b} = 0.
			\end{equation}
		\end{Property}

		\begin{Proof}
			\label{prf_EEq}
			考虑时空 $(M,\tensor{g}{_a_b})$ 及 $M$ 上的一族度规 $\tensor{g}{_a_b}(\lambda)$,满足 $\tensor{g}{_a_b}(0)=\tensor{g}{_a_b}$,则任何依赖度规的量 $T$ 的变分为
			\begin{equation}
				\var T\left( \tensor{g}{_a_b} \right) \definedby \left. \dv{T\left( \tensor{g}{_a_b}(\lambda) \right)}{\lambda}  \right|_{\lambda=0}.
			\end{equation}

			注意到
			\begin{equation}
				\var{\Lad} = \underbrace{\sqrt{-g} \tensor{g}{^a^b} \var{\tensor{R}{_a_b}}}_{\RomanNumeralCaps{1}} + \underbrace{\sqrt{-g} \tensor{R}{_a_b} \var{\tensor{g}{^a^b}}}_{\RomanNumeralCaps{2}} + \underbrace{R \var{\sqrt{-g}}}_{\RomanNumeralCaps{3}},\label{eq-varL_EH}
			\end{equation}
			首先考虑与 $\tensor{g}{_a_b}(\lambda)$ 适配的导数算符 $\tensor{\myvaried{\nabla}}{_a}$,假定与 $\Nabla{a}$ 相差 $\tensor{C}{^c_a_b}(\lambda)$,即
			\begin{equation}
				\left( \Nabla{a} - \tensor{\myvaried{\nabla}}{_a} \right) \tensor{\omega}{_b} = \tensor{C}{^c_a_b}(\lambda) \tensor{\omega}{_c},
			\end{equation}
			通过 $\tensor{\myvaried{\nabla}}{_a} \tensor{g}{_b_c}(\lambda) = 0$ ,与计算克氏符类似,可算得
			\begin{equation}
				\tensor{C}{^c_a_b}(\lambda) = \frac{1}{2} \tensor{g}{^c^d}(\lambda) \left( \Nabla{a} \tensor{g}{_b_d}(\lambda) + \Nabla{b} \tensor{g}{_a_d}(\lambda) - \Nabla{d} \tensor{g}{_a_b}(\lambda) \right),\label{eq-Clambda}
			\end{equation}
			进而考虑 $\tensor{\myvaried{\nabla}}{_a}$ 相应的黎曼张量 $\tensor{R}{_a_b_c^d}(\lambda)$,按照定义,有
			\begin{equation}
				\begin{split}
					\tensor{R}{_a_b_c^d}(\lambda) \tensor{\omega}{_d} ={}& 2 \tensor{\myvaried{\nabla}}{_{[a}} \tensor{\myvaried{\nabla}}{_{b]}} \tensor{\omega}{_c}\\
					={}& \tensor{\myvaried{\nabla}}{_a} \left( \Nabla{b} \tensor{\omega}{_c} - \tensor{C}{^d_b_c}(\lambda) \tensor{\omega}{_d} \right) - \tensor{\myvaried{\nabla}}{_b} \left( \Nabla{a} \tensor{\omega}{_c} - \tensor{C}{^d_a_c}(\lambda) \tensor{\omega}{_d} \right)\\
					={}& \left( {\color{red}\Nabla{a} \Nabla{b}\tensor{\omega}{_c}} - {\color{blue}\tensor{C}{^d_a_b}(\lambda) \Nabla{d} \tensor{\omega}{_c}} - \tensor{C}{^d_a_c}(\lambda) \Nabla{b} \tensor{\omega}{_d} \right)\\
					& {}- \left[ \Nabla{a} \left( \tensor{C}{^d_b_c}(\lambda) \tensor{\omega}{_d} \right) - {\color{green}\tensor{C}{^e_a_b}(\lambda) \tensor{C}{^d_e_c}(\lambda) \tensor{\omega}{_d}} - \tensor{C}{^e_a_c}(\lambda) \tensor{C}{^d_b_e}(\lambda) \tensor{\omega}{_d} \right]\\
					& {}- \left( {\color{red}\Nabla{b} \Nabla{a}\tensor{\omega}{_c}} - {\color{blue}\tensor{C}{^d_b_a}(\lambda) \Nabla{d} \tensor{\omega}{_c}} - \tensor{C}{^d_b_c}(\lambda) \Nabla{a} \tensor{\omega}{_d} \right)\\
					& {} + \left[ \Nabla{b} \left( \tensor{C}{^d_a_c}(\lambda) \tensor{\omega}{_d} \right) - {\color{green}\tensor{C}{^e_b_a}(\lambda) \tensor{C}{^d_e_c}(\lambda) \tensor{\omega}{_d}} - \tensor{C}{^e_b_c}(\lambda) \tensor{C}{^d_a_e}(\lambda) \tensor{\omega}{_d} \right]\\
					={} & {\color{red} \tensor{R}{_a_b_c^d} \tensor{\omega}{_d}} - 2 \left( \Nabla{{[a|}} \tensor{C}{^d_{b]}_c}(\lambda) \right) \tensor{\omega}{_d} + 2 \tensor{C}{^e_{[a}_{|c|}}(\lambda) \tensor{C}{^d_{b]}_e}(\lambda) \tensor{\omega}{_d},
				\end{split}
			\end{equation}
			即
			\begin{gather}
				\tensor{R}{_a_b_c^d}(\lambda) = \tensor{R}{_a_b_c^d} - 2 \Nabla{{[a}} \tensor{C}{^d_{b]}_c}(\lambda) + 2 \tensor{C}{^e_c_{[a}}(\lambda) \tensor{C}{^d_{b]}_e}(\lambda),\\
				\tensor{R}{_a_b}(\lambda) = \tensor{R}{_a_c_b^c}(\lambda) = \tensor{R}{_a_b} - 2 \Nabla{{[a}} \tensor{C}{^c_{c]}_b}(\lambda) + 2 \tensor{C}{^e_b_{[a}}(\lambda) \tensor{C}{^c_{c]}_e}(\lambda),
			\end{gather}
			由于 $\tensor{C}{^c_a_b}(0)=0$,求导得
			\begin{equation}
				\var{\tensor{R}{_a_b}} = - 2 \Nabla{{[a}} \var \tensor{C}{^c_{c]}_b} = \Nabla{c} \var \tensor{C}{^c_a_b} - \Nabla{a} \var \tensor{C}{^c_c_b},\label{eq-varR_varC}
			\end{equation}
			而由~\eqref{eq-Clambda}得
			\begin{equation}
				\begin{split}
					\var{\tensor{C}{^c_a_b}} &= \frac{1}{2} \var\tensor{g}{^c^d} \left( \Nabla{a} \tensor{g}{_b_d} + \Nabla{b} \tensor{g}{_a_d} - \Nabla{d} \tensor{g}{_a_b} \right) + \frac{1}{2} \tensor{g}{^c^d} \left( \Nabla{a} \var \tensor{g}{_b_d} + \Nabla{b} \var \tensor{g}{_a_d} - \Nabla{d} \var \tensor{g}{_a_b} \right)\\
					&= \frac{1}{2} \tensor{g}{^c^d} \left( \Nabla{a} \var \tensor{g}{_b_d} + \Nabla{b} \var \tensor{g}{_a_d} - \Nabla{d} \var \tensor{g}{_a_b} \right),\\
					\var{\tensor{C}{^c_c_b}} &= \frac{1}{2} \tensor{g}{^c^d} \left( \Nabla{c} \var \tensor{g}{_b_d} + \Nabla{b} \var \tensor{g}{_c_d} - \Nabla{d} \var \tensor{g}{_c_b} \right)\\
					&= \frac{1}{2} \tensor{g}{^c^d} \Nabla{b} \var \tensor{g}{_c_d},
				\end{split}
			\end{equation}
			代入~\eqref{eq-varR_varC} 得
			\begin{equation}
				\begin{split}
					\var{\tensor{R}{_a_b}} &= \frac{1}{2} \tensor{g}{^c^d} \Nabla{c} \left( \Nabla{a} \var \tensor{g}{_b_d} + \Nabla{b} \var \tensor{g}{_a_d} - \Nabla{d} \var \tensor{g}{_a_b} \right) - \frac{1}{2} \tensor{g}{^c^d} \Nabla{a} \Nabla{b} \var \tensor{g}{_c_d}\\
					&= \frac{1}{2} \tensor{g}{^c^d} \left( \Nabla{c} \Nabla{a} \var \tensor{g}{_b_d} + \Nabla{c} \Nabla{b} \var \tensor{g}{_a_d} - \Nabla{c} \Nabla{d} \var \tensor{g}{_a_b} - \Nabla{a} \Nabla{b} \var \tensor{g}{_c_d} \right),
				\end{split}
			\end{equation}
			则~\eqref{eq-varL_EH} 中的 \RomanNumeralCaps{1} 为
			\begin{equation}
				\begin{split}
					\RomanNumeralCaps{1} &= \sqrt{-g} \tensor{g}{^a^b} \var{\tensor{R}{_a_b}}\\
					&= \frac{1}{2} \sqrt{-g} \left( \tensor{\nabla}{^d} \tensor{\nabla}{^b} \var \tensor{g}{_b_d} + \tensor{\nabla}{^d} \tensor{\nabla}{^a} \var \tensor{g}{_a_d} - \tensor{g}{^a^b} \tensor{\nabla}{^d} \Nabla{d} \var \tensor{g}{_a_b} - \tensor{g}{^c^d} \tensor{\nabla}{^a} \Nabla{a} \var \tensor{g}{_c_d} \right)\\
					&= \sqrt{-g} \left( \tensor{\nabla}{^a} \tensor{\nabla}{^b} \var{\tensor{g}{_a_b}} - \tensor{g}{^b^c} \tensor{\nabla}{^a} \Nabla{a} \var{\tensor{g}{_b_c}} \right)\\
					&= \sqrt{-g} \Nabla{a} \tensor{v}{^a},
				\end{split}
			\end{equation}
			其中
			\begin{equation}
				\tensor{v}{^a} \definedby \tensor{\nabla}{^b} \var{\tensor{g}{_a_b}} - \tensor{g}{^b^c} \Nabla{a} \var{\tensor{g}{_b_c}},
			\end{equation}
			故这一项仅为边界项。

			对 $\tensor{\delta}{^a_b} = \tensor{g}{^a^c} \tensor{g}{_c_b}$ 两边变分得关系
			\begin{equation}
				\begin{split}
					\var{\tensor{g}{_a_b}} = - \tensor{g}{_a_c} \tensor{g}{_b_d} \var{\tensor{g}{^c^d}} \qc \var{\tensor{g}{^a^b}} = - \tensor{g}{^a^c} \tensor{g}{^b^d} \var{\tensor{g}{_c_d}},
				\end{split}
			\end{equation}
			故得~\eqref{eq-varL_EH} 中的 \RomanNumeralCaps{2} 为
			\begin{equation}
				\begin{split}
					\RomanNumeralCaps{2} &= \sqrt{-g} \tensor{R}{_a_b} \var{\tensor{g}{^a^b}}\\
					&= - \sqrt{-g} \tensor{R}{_a_b} \tensor{g}{^a^c} \tensor{g}{^b^d} \var{\tensor{g}{_c_d}}\\
					&= - \sqrt{g} \tensor{R}{^a^b} \var{\tensor{g}{_a_b}},
				\end{split}
			\end{equation}
			最后考虑 \RomanNumeralCaps{3},注意到若记 $\tensor{\nvol}{_a_b_c_d}$ 为坐标体元,有行列式表达式
			\begin{equation}
				g = \frac{1}{4!} \tensor{\nvol}{^a^b^c^d} \tensor{\nvol}{^e^f^g^h} \tensor{g}{_a_e} \tensor{g}{_b_f} \tensor{g}{_c_g} \tensor{g}{_d_h},
			\end{equation}
			则
			\begin{equation}
				\begin{split}
					\var g &= \frac{1}{3!} \tensor{\nvol}{^a^b^c^d} \tensor{\nvol}{^e^f^g^h} \tensor{g}{_a_e} \tensor{g}{_b_f} \tensor{g}{_c_g} \var \tensor{g}{_d_h},
				\end{split}
			\end{equation}
			记
			\begin{equation}
				\tensor{T}{^d^h} \definedby \frac{1}{3!} \tensor{\nvol}{^a^b^c^d} \tensor{\nvol}{^e^f^g^h} \tensor{g}{_a_e} \tensor{g}{_b_f} \tensor{g}{_c_g},
			\end{equation}
			则显然它对称,迹为 $4g$,有
			\begin{equation}
				\tensor{T}{^d^h} = g \tensor{g}{^d^h} + \tensor{S}{^d^h},
			\end{equation}
			其中 $\tensor{S}{^d^h}$ 无迹。另一方面,
			\begin{equation}
				\begin{split}
					\tensor{T}{^d^h} \tensor{g}{_h_l} &= \frac{1}{3!} \tensor{\nvol}{^a^b^c^d} \tensor{\nvol}{^e^f^g^h} \tensor{g}{_a_e} \tensor{g}{_b_f} \tensor{g}{_c_g} \tensor{g}{_h_l}\\
					&= \frac{1}{3!} \tensor{\nvol}{^\mu^\nu^\sigma^\rho} \tensor{\nvol}{^\alpha^\beta^\gamma^\delta} \tensor{g}{_\mu_\alpha} \tensor{g}{_\nu_\beta} \tensor{g}{_\sigma_\gamma} \tensor{g}{_\delta_\eta} \tensor{\left( \pdv{x^\rho} \right)}{^d} \tensor{\left( \dd{x^\eta} \right)}{_l}\\
					&= \tensor{\nvol}{^0^1^2^\rho} \tensor{\nvol}{^\alpha^\beta^\gamma^\delta} \tensor{g}{_0_\alpha} \tensor{g}{_1_\beta} \tensor{g}{_2_\gamma} \tensor{g}{_\delta_\eta} \tensor{\left( \pdv{x^\rho} \right)}{^d} \tensor{\left( \dd{x^\eta} \right)}{_l}\\
					&= \tensor{\nvol}{^\alpha^\beta^\gamma^\delta} \tensor{g}{_0_\alpha} \tensor{g}{_1_\beta} \tensor{g}{_2_\gamma} \tensor{g}{_\delta_3} \tensor{\left( \pdv{x^3} \right)}{^d} \tensor{\left( \dd{x^3} \right)}{_l}\\
					&= g\tensor{\delta}{^d_l},
				\end{split}
			\end{equation}
			其中倒数第二行中 $\eta$ 必取 $3$ 是因为,否则,不妨设 $\eta$ 取 $0$,则 $\alpha$ 与 $\delta$ 对称,与 $\tensor{\nvol}{^\alpha^\beta^\gamma^\delta}$ 缩并为 $0$。则可知 $\tensor{S}{^d^h}=0$,
			\begin{equation}
				\tensor{T}{^d^h} = \frac{1}{3!} \tensor{\nvol}{^a^b^c^d} \tensor{\nvol}{^e^f^g^h} \tensor{g}{_a_e} \tensor{g}{_b_f} \tensor{g}{_c_g} = g \tensor{g}{^d^h},
			\end{equation}
			故
			\begin{equation}
				\var{g} = g \tensor{g}{^a^b} \var{\tensor{g}{_a_b}},
			\end{equation}
			于是
			\begin{equation}
				\begin{split}
					\RomanNumeralCaps{3} &= R \var{\sqrt{-g}}\\
					&= - \frac{R}{2\sqrt{-g}} \var{g}\\
					&= \frac{1}{2} \sqrt{-g} R \tensor{g}{^a^b} \var{\tensor{g}{_a_b}},
				\end{split}
			\end{equation}
			故
			\begin{equation}
				\var{\Lad} = \RomanNumeralCaps{1} + \RomanNumeralCaps{2} + \RomanNumeralCaps{3} = \sqrt{-g} \Nabla{a} \tensor{v}{^a} - \sqrt{-g} \left( \tensor{R}{^a^b} - \frac{1}{2} R \tensor{g}{^a^b} \right) \var{\tensor{g}{_a_b}},
			\end{equation}
			可得真空场方程。{\normalfont\ttfamily\color{green} 点击返回~\eqref{eq-EEq}。}
		\end{Proof}

		\begin{Remark}
			\begin{enumerate}
				\item 也可采用矩阵的语言,使用伴随矩阵为工具计算 $\delta g$,参见\inlinecite{liang3}。
				\item 也可考察 $\La = R \form{\vol}$,则需要计算适配体元的变分。梁灿彬老师在 \inlinecite{liang3} 的下册第~9~页中写道
				\begin{quote}
					……对 $\tensor{g}{_a_b}$ 变分时就必须考虑 $\form{\vol}$ 的相应变分,从而给计算带来麻烦。
				\end{quote}
				进而得出用标量密度 $\Lad$ 更合适的结论。然而鄙人实际算了发现没觉得体元的变分有多复杂……见下文。

				记 $\form{\vol}(\lambda)$ 是与 $\tensor{g}{_a_b}(\lambda)$ 适配的体元,则
				\begin{equation}
					\tensor{g}{^a^e}(\lambda) \tensor{g}{^b^f}(\lambda) \tensor{g}{^c^g}(\lambda) \tensor{g}{^d^h}(\lambda) \tensor{\vol}{_a_b_c_d}(\lambda) \tensor{\vol}{_e_f_g_h}(\lambda) = - 4!,
				\end{equation}
				对 $\lambda$ 求导得
				\begin{equation}
					\begin{split}
						0 &= 4 \left( \var \tensor{g}{^a^e} \right) \tensor{g}{^b^f} \tensor{g}{^c^g} \tensor{g}{^d^h} \tensor{\vol}{_a_b_c_d} \tensor{\vol}{_e_f_g_h} + 2 \tensor{g}{^a^e} \tensor{g}{^b^f} \tensor{g}{^c^g} \tensor{g}{^d^h} \tensor{\vol}{_a_b_c_d} \var \tensor{\vol}{_e_f_g_h}\\
						&= - 4 \times 3! \times \tensor{g}{_a_e} \var{\tensor{g}{^a^e}} + 2 \tensor{\vol}{^e^f^g^h} \var{\tensor{\vol}{_e_f_g_h}},
					\end{split}
				\end{equation}
				故
				\begin{equation}
					\begin{split}
						\tensor{\vol}{^a^b^c^d} \var{\tensor{\vol}{_a_b_c_d}} &= - 2 \times 3! \times \tensor{g}{^a^b} \var{\tensor{g}{_a_b}}\\
						&= - \frac{1}{2} 4! \tensor{g}{^a^b} \var{\tensor{g}{_a_b}}\\
						&= \frac{1}{2} \tensor{\vol}{^e^f^g^h} \left( \tensor{\vol}{_e_f_g_h} \tensor{g}{^a^b} \var{\tensor{g}{_a_b}} \right),
					\end{split}
				\end{equation}
				两边取对偶形式去掉 $\form{\vol}$,知
				\begin{equation}
					\var{\tensor{\vol}{_a_b_c_d}} = \frac{1}{2} \tensor{\vol}{_a_b_c_d} \tensor{g}{^e^f} \var{\tensor{g}{_e_f}}.
				\end{equation}

				算完适配体元变分后,即可算得
				\begin{equation}
					\begin{split}
						\var{\La} &= \tensor{g}{^a^b} \var(\tensor{R}{_a_b}) \form{\vol} - \tensor{R}{^a^b} \var(\tensor{g}{_a_b}) \form{\vol} + R \var{\form{\vol}}\\
						&= \left( \Nabla{a} \tensor{v}{^a} \right) \form{\vol} - \left( \tensor{R}{^a^b} - \frac{1}{2} R \tensor{g}{^a^b} \right) \var{\tensor{g}{_a_b}} \form{\vol},
					\end{split}
				\end{equation}
				从而得到场方程。
			\end{enumerate}
			{\normalfont\ttfamily\color{green} 点击返回~\eqref{eq-EEq}。}
		\end{Remark}

		\begin{Property}[\pageref{pro-KLnh} 页命题~\ref{pro-KLnh}]
			\begin{equation}
				\tensor{K}{_a_b} = \frac{1}{2} \Ld{n} \tensor{h}{_a_b},
			\end{equation}
			其中 $\Ld{n}$ 表示沿 $\tensor{n}{^a}$ 的李导数。
		\end{Property}
		\begin{Proof}
			\label{prf-KLnh}
			由李导数公式
			\begin{equation}
				\Ld{v} \tensor{T}{^{a_1 \cdots a_k}_{b_1 \cdots b_l}} = \tensor{v}{^c} \Nabla{c} \tensor{T}{^{a_1 \cdots a_k}_{b_1 \cdots b_l}} - \sum_{i=1}^k \tensor{T}{^{a_1 \cdots c \cdots a_k}_{b_1 \cdots b_l}} \Nabla{c} \tensor{v}{^{a_i}} + \sum_{j=1}^l \tensor{T}{^{a_1 \cdots a_k}_{b_1 \cdots c \cdots b_l}} \Nabla{{b_j}} \tensor{v}{^c}
			\end{equation}
			知
			\begin{equation}
				\Ld{v} \tensor{g}{_a_b} = 2 \Nabla{{(a}} \tensor{v}{_{b)}}
			\end{equation}
			及
			\begin{equation}
				\Ld{n} \tensor{n}{_a} = \tensor{n}{^b} \Nabla{b} \tensor{n}{_a} + \tensor{n}{_b} \Nabla{a} \tensor{n}{^b} = \tensor{n}{^b} \Nabla{b} \tensor{n}{_a},
			\end{equation}
			故
			\begin{equation}
				\begin{split}
					\Ld{n} \tensor{h}{_a_b} &= \Ld{n} \left( \tensor{g}{_a_b} + \tensor{n}{_a} \tensor{n}{_b} \right)\\
					&= 2 \Nabla{{(a}} \tensor{n}{_{b)}} + 2 \tensor{n}{_{(a}} \Ld{n} \tensor{n}{_{b)}}\\
					&= 2 \Nabla{{(a}} \tensor{n}{_{b)}} + 2 \tensor{n}{_{(a}} \tensor{n}{^c} \Nabla{{|c|}} \tensor{n}{_{b)}}\\
					&= 2 \tensor{\delta}{^c_{(a}} \Nabla{{|c|}} \tensor{n}{_b} + 2 \tensor{n}{_{(a}} \tensor{n}{^c} \Nabla{{|c|}} \tensor{n}{_{b)}}\\
					&= 2 \tensor{K}{_{(ab)}}\\
					&= 2 \tensor{K}{_a_b},
				\end{split}
			\end{equation}
			证毕。{\normalfont\ttfamily\color{green} 点击返回命题~\ref{pro-KLnh}。}
		\end{Proof}

		\begin{Property}[\pageref{eq-L_split}页\eqref{eq-L_split}]
			$\Lad_{\text{EH}} = \frac{1}{2\gkappa} \sqrt{- \det g} \curR$ 用空间量表示为
			\begin{equation}
				\Lad = \frac{1}{2\gkappa} \sqrt{h} N \left( \spacecurR - K^2 + \tensor{K}{_a_b} \tensor{K}{^a^b} \right).
			\end{equation}
			
		\end{Property}
		\begin{Proof}
			\label{ap-eq-L_split}
			首先,先证明:
			\begin{Lemma}
				\begin{equation}
					\sqrt{-g} = N\sqrt{h}.
				\end{equation}
				\begin{Proof}
					注意到3维的坐标体元为
					\begin{equation}
						\begin{split}
							\tensor{\nvol}{_b_c_d} &= \frac{1}{\sqrt{-g}} \left( {\tensor{t}{^a} \tensor{\vol}{_a_b_c_d}} \right)\sptilde\\
							&= \frac{1}{\sqrt{-g}} \left( {\left( N \tensor{n}{^a} + \tensor{N}{^a} \right) \tensor{\vol}{_a_b_c_d}} \right)\sptilde\\
							&= \frac{1}{\sqrt{-g}} N \tensor{\vol}{_b_c_d} + 0\\
							&= \frac{\sqrt{h}}{\sqrt{-g}} N \tensor{\nvol}{_b_c_d},
						\end{split}
					\end{equation}
					故
					\begin{equation}
						\sqrt{-g} = N\sqrt{h},
					\end{equation}
					其中 $\sptilde$ 表示取空间投影,而 $\tensor{N}{^a} \tensor{\vol}{_a_b_c_d}$ 的空间投影为零可通过如下考虑看出:任取空间面上的基底 $\tensor{e}{^a_i}$,$i=1,2,3$,注意到 $\left( \tensor{N}{^a} \tensor{\vol}{_a_b_c_d} \right)\sptilde$ 独立分量只有一个,不妨考虑 $\tensor{N}{^a} \tensor{\vol}{_a_b_c_d} \tensor{e}{^a_1} \tensor{e}{^a_2} \tensor{e}{^a_3}$,但 $\tensor{N}{^a}$ 是 $\left\{ \tensor{e}{^a_i} \right\}_{i=1}^3$ 的线性组合,故分量为零。
				\end{Proof}
			\end{Lemma}

			接下来分解曲率,借用高斯方程~\eqref{eqgauss}:
			\begin{equation*}
				\tensor{\spacecurR}{_a_b_c^d} = \tensor{h}{_a^k} \tensor{h}{_b^l} \tensor{h}{_c^m} \tensor{h}{_n^d} \tensor{\curR}{_k_l_m^n} - 2 \tensor{K}{_{c[a}} \tensor{K}{_{b]}^d},
			\end{equation*}
			注意到
			\begin{equation*}
				R = \tensor{g}{^a^b} \tensor{R}{_a_b} = \left( \tensor{h}{^a^b} - \tensor{n}{^a}\tensor{n}{^b} \right) \tensor{R}{_a_b},
			\end{equation*}
			考虑 Ricci 张量的时时分量:
			\begin{equation*}
				\begin{split}
					\tensor{R}{_a_b} \tensor{n}{^a} \tensor{n}{^b} &= \tensor{R}{_a_c_b^c} \tensor{n}{^a} \tensor{n}{^b}\\
					&= - \tensor{n}{^a} \left( \Nabla{a} \Nabla{c} - \Nabla{c} \Nabla{a} \right) \tensor{n}{^c}\\
					&= - \Nabla{a} \left( \tensor{n}{^a} \Nabla{c} \tensor{n}{^c} \right) + \left( \Nabla{a} \tensor{n}{^a} \right) \Nabla{c} \tensor{n}{^c} + \Nabla{c} \left( \tensor{n}{^a} \Nabla{a} \tensor{n}{^c} \right) - \left( \Nabla{c} \tensor{n}{^a} \right) \left( \Nabla{a} \tensor{n}{^c} \right)\\
					&= - \Nabla{a} \left( \tensor{n}{^a} \Nabla{c} \tensor{n}{^c} \right) + K^2 + \Nabla{c} \left( \tensor{n}{^a} \Nabla{a} \tensor{n}{^c} \right) - \tensor{K}{_a_c} \tensor{K}{^a^c},
				\end{split}
			\end{equation*}
			而
			\begin{equation*}
				\begin{split}
					\tensor{h}{^a^b} \tensor{R}{_a_b} &= \tensor{h}{^a^b} \tensor{g}{^c^d} \tensor{R}{_a_c_b_d}\\
					&= \tensor{h}{^a^b} \left( \tensor{h}{^c^d} - \tensor{n}{^c} \tensor{n}{^d} \right) \tensor{R}{_a_c_b_d}\\
					&= \tensor{h}{^a^b} \tensor{h}{^c^d} \left( \tensor{\spacecurR}{_a_c_b_d} + 2 \tensor{K}{_{b[a}} \tensor{K}{_{c]d}} \right) - \left( \tensor{g}{^a^b} + \tensor{n}{^a} \tensor{n}{^b} \right) \tensor{n}{^c} \tensor{n}{^d} \tensor{R}{_a_c_b_d}\\
					&= \spacecurR + K^2 - \tensor{K}{_a_b} \tensor{K}{^a^b} - \tensor{n}{^c} \tensor{n}{^d} \tensor{R}{_c_d} - 0,
				\end{split}
			\end{equation*}
			故
			\begin{equation*}
				\begin{split}
					R &= \left( \tensor{h}{^a^b} - \tensor{n}{^a}\tensor{n}{^b} \right) \tensor{R}{_a_b}\\
					&= \spacecurR + K^2 + \tensor{K}{_a_b} \tensor{K}{^a^b} - 2 \tensor{n}{^a} \tensor{n}{^b} \tensor{R}{_a_b}\\
					&= \spacecurR - K^2 + \tensor{K}{_a_b} \tensor{K}{^a^b} + 2 \Nabla{a} \left( \tensor{n}{^a} \Nabla{b} \tensor{n}{^b} - \tensor{n}{^b} \Nabla{b} \tensor{n}{^a} \right),
				\end{split}
			\end{equation*}
			弃去边界项,命题得证。
			{\normalfont\ttfamily\color{green} 点击返回~\eqref{eq-L_split}。}
		\end{Proof}

		\begin{Property}[\pageref{eq-pi}页\eqref{eq-pi}]
			可求得共轭动量
			\begin{gather}
				\pi_N = \pdv{\Lad}{\dot{N}} = 0 \qc \tensor{\pi}{^a} = \pdv{\Lad}{\tensor{\dot{N}}{_a}} = 0, \label{ap-eq-constrain12}\\
				\tensor{\pi}{^a^b} = \pdv{\Lad}{\tensor{\dot{h}}{_a_b}} = \frac{1}{2\gkappa} \sqrt{h} \left( \tensor{K}{^a^b} - K \tensor{h}{^a^b} \right), \label{ap-eq-pi}
			\end{gather}
			其中 $\pi_N$, $\tensor{\pi}{^a}$, $\tensor{\pi}{^a^b}$ 分别是与 $N$, $\tensor{N}{_a}$, $\tensor{h}{_a_b}$ 共轭的动量。
		\end{Property}
		
		\begin{Proof}
			\label{ap-eq-pi}
			首先,易发现 $\Lad$ 不含 $\dot{N}$ 和 $\tensor{\dot{N}}{^a}$,故~\eqref{ap-eq-constrain12} 显然。而依赖 $\tensor{\dot{h}}{_a_b}$ 的项只有 $\tensor{K}{_a_b}$,注意到~\eqref{eq-K_doth}:
			\begin{equation*}
				\tensor{K}{_a_b} = \frac{1}{2N} \left( \tensor{\dot{h}}{_a_b} - 2 \tensor{D}{_{(a}} \tensor{N}{_{b)}} \right),
			\end{equation*}
			得
			\begin{equation}
				\begin{split}
					\tensor{\pi}{^a^b} &= \pdv{\Lad}{\tensor{\dot{h}}{_a_b}}\\
					&= \pdv{\Lad}{\tensor{K}{_c_d}} \pdv{\tensor{K}{_c_d}}{\tensor{\dot{h}}{_a_b}}\\
					&= \frac{1}{2\gkappa} \sqrt{h} N \left( - 2 K \tensor{h}{^c^d} + 2 \tensor{K}{^c^d} \right) \times \frac{1}{2N} \tensor{\delta}{^a_{(c}} \tensor{\delta}{^b_{d)}}\\
					&= \frac{1}{2\gkappa} \sqrt{h} \left( \tensor{K}{^a^b} - K \tensor{h}{^a^b} \right),
				\end{split}
			\end{equation}
			证毕。
			{\normalfont\ttfamily\color{green} 点击返回~\eqref{eq-pi}。}
		\end{Proof}

		\begin{Property}[\pageref{eq-ADM_H}页 \eqref{eq-ADM_H}]
			ADM形式的哈密顿量为
			\begin{equation}
				H[N,\tensor{N}{_a}, \tensor{h}{_a_b}, \tensor{\pi}{^a^b}] = \int_{\spc} \dd[3]{x} \left( N C + \tensor{N}{_a} \tensor{V}{^a} \right), \label{ap-eq-ADM_H}
			\end{equation}
			其中
			\begin{gather}
				C \definedby - \frac{\sqrt{h}}{2\gkappa} \spacecurR + \frac{2\gkappa}{\sqrt{h}} \left( \tensor{\pi}{_a_b} \tensor{\pi}{^a^b} - \frac{1}{2} \pi^2 \right),\\
				\tensor{V}{^a} \definedby -2 \spaceD{b} \tensor{\pi}{^a^b}.
			\end{gather}
		\end{Property}

		\begin{Proof}
			\label{ap-eq-ADM_H}
			由~\eqref{ap-eq-pi} 取迹得
			\begin{equation}
				\pi = \frac{1}{2\gkappa} \sqrt{h} \left( K - 3 K \right) = - \frac{1}{\gkappa} \sqrt{h} K,
			\end{equation}
			代入反解得
			\begin{equation}
				\begin{split}
					\tensor{K}{^a^b} &= \frac{2\gkappa}{\sqrt{h}} \tensor{\pi}{^a^b} + K \tensor{h}{^a^b}\\
					&= \frac{2\gkappa}{\sqrt{h}} \tensor{\pi}{^a^b} - \frac{\gkappa}{\sqrt{h}} \pi \tensor{h}{^a^b}\\
					&= \frac{2\gkappa}{\sqrt{h}} \left( \tensor{\pi}{^a^b} - \frac{1}{2} \pi \tensor{h}{^a^b} \right),
				\end{split}
			\end{equation}
			故
			\begin{equation}
				\begin{split}
					\Had ={}& \tensor{\dot{h}}{_a_b} \tensor{\pi}{^a^b} - \Lad\\
					={}& \left( 2 N \tensor{K}{_a_b} + 2 \tensor{D}{_{(a}} \tensor{N}{_{b)}} \right) \tensor{\pi}{^a^b} - \frac{1}{2\gkappa} \sqrt{h} N \left( \spacecurR - K^2 + \tensor{K}{_a_b} \tensor{K}{^a^b} \right)\\
					={}& \frac{4\gkappa}{\sqrt{h}} N \left( \tensor{\pi}{_a_b} - \frac{1}{2} \pi \tensor{h}{_a_b} \right) \tensor{\pi}{^a^b} + 2 \tensor{\pi}{^a^b} \tensor{D}{_a} \tensor{N}{_b}\\
					& {} - \frac{\sqrt{h}}{2\gkappa} N \left[ \spacecurR - \frac{\gkappa^2}{h} \pi^2 + \frac{4\gkappa^2}{h} \left( \tensor{\pi}{^a^b} - \frac{1}{2} \pi \tensor{h}{^a^b} \right) \left( \tensor{\pi}{_a_b} - \frac{1}{2} \pi \tensor{h}{_a_b} \right) \right]\\
					={} & \frac{4\gkappa}{\sqrt{h}} N \left( \tensor{\pi}{_a_b} \tensor{\pi}{^a^b} - \frac{1}{2} \pi^2 \right) + 2 \tensor{D}{_a} \left( \tensor{\pi}{^a^b} \tensor{N}{_b} \right) - 2 \tensor{N}{_b} \tensor{D}{_a} \tensor{\pi}{^a^b}\\
					& {} - N \left[ \frac{\sqrt{h}}{2\gkappa} \spacecurR - \frac{\gkappa}{2\sqrt{h}} \pi^2 + \frac{2\gkappa}{\sqrt{h}} \left( \tensor{\pi}{_a_b} \tensor{\pi}{^a^b} - \pi^2 + \frac{3}{4} \pi^2 \right) \right]\\
					={} & 2 \tensor{D}{_a} \left( \tensor{\pi}{^a^b} \tensor{N}{_b} \right) - 2 \tensor{N}{_a} \tensor{D}{_b} \tensor{\pi}{^a^b} - \frac{\sqrt{h}}{2\gkappa} N \spacecurR + \frac{2\gkappa}{\sqrt{h}} N \left( \tensor{\pi}{_a_b} \tensor{\pi}{^a^b} - \frac{1}{2} \pi^2 \right),
				\end{split}
			\end{equation}
			弃去第一项边界项,得
			\begin{equation}
				\Had = N C + \tensor{N}{_a} \tensor{V}{^a},
			\end{equation}
			其中
			\begin{gather}
				C \definedby - \frac{\sqrt{h}}{2\gkappa} \spacecurR + \frac{2\gkappa}{\sqrt{h}} \left( \tensor{\pi}{_a_b} \tensor{\pi}{^a^b} - \frac{1}{2} \pi^2 \right),\\
				\tensor{V}{^a} \definedby -2 \spaceD{b} \tensor{\pi}{^a^b}.
			\end{gather}
			{\normalfont\ttfamily\color{green} 点击返回~\eqref{eq-ADM_H}。}
		\end{Proof}

		\begin{Property}[\pageref{eq-ADM_constrain_alg}页 \eqref{eq-ADM_constrain_alg}]
			ADM 形式中,约束之间的泊松括号为
			\begin{equation}
				\begin{split}
					\left\{ V({u}), V({v}) \right\} &= 2\gkappa V(\Ld{{u}} {v}),\\
					\left\{ V({v}), C(f) \right\} &= 2\gkappa C(v(f)),\\
					\left\{ C(f), C(f') \right\} &= 2\gkappa V(f\tensor{D}{^a}f'-f' \tensor{D}{^a}f).
				\end{split}
			\end{equation}
		\end{Property}

		\begin{Proof}
			\label{ap-eq-ADM_constrain_alg}
			利用
			\begin{gather}
				\var(\sqrt{h} \spacecurR) = \sqrt{h} \left( \tensor{\spacecurR}{_a_b} - \frac{1}{2} \spacecurR \tensor{h}{_a_b} \right) \var{\tensor{h}{^a^b}} + \sqrt{h} \tensor{D}{^a} \left( \tensor{D}{^b} \var{\tensor{h}{_a_b}} - \tensor{h}{^c^d} \tensor{D}{_a} \var{\tensor{h}{_c_d}} \right),\\
				\var h = h \tensor{h}{^a^b} \var \tensor{h}{_a_b},\\
				\var{\tensor{h}{^a^b}} = - \tensor{h}{^a^c} \tensor{h}{^b^d} \var{\tensor{h}{_c_d}},
			\end{gather}
			标量约束的变分
			\begin{equation}
				\var C(f) = \underbrace{- \frac{1}{2\gkappa} \var \int_{\spc} \dd[3]{x} \sqrt{h} \spacecurR f}_{\mathrm{I}} + \underbrace{2\gkappa \var \int_{\spc} \frac{f}{h} \left( \tensor{\pi}{_a_b} \tensor{\pi}{^a^b} - \frac{1}{2} \pi^2 \right)}_{\mathrm{II}},
			\end{equation}
			分别计算:
			\begin{align*}
				\mathrm{I} &= - \frac{1}{2\gkappa} \int_{\spc} f \left[ \left( \tensor{\spacecurR}{_a_b} - \frac{1}{2} \spacecurR \tensor{h}{_a_b} \right) \var{\tensor{h}{^a^b}} + \tensor{D}{^a} \left( \tensor{D}{^b} \var{\tensor{h}{_a_b}} - \tensor{h}{^c^d} \tensor{D}{_a} \var{\tensor{h}{_c_d}} \right) \right]\\
				&= - \frac{1}{2\gkappa} \int_{\spc} -f \left[ \left( \tensor{\spacecurR}{^a^b} - \frac{1}{2} \spacecurR \tensor{h}{^a^b} \right) \var{\tensor{h}{_a_b}} \right] - \left( \tensor{D}{^b} \var{\tensor{h}{_a_b}} - \tensor{h}{^c^d} \tensor{D}{_a} \var{\tensor{h}{_c_d}} \right) \tensor{D}{^a} f\\ \displaybreak[1]
				&= \frac{1}{2\gkappa} \int_{\spc} \left[ f \left( \tensor{\spacecurR}{^a^b} - \frac{1}{2} \spacecurR \tensor{h}{^a^b} \right) - \left( \tensor{D}{^b} \tensor{D}{^a} f - \tensor{h}{^a^b} \tensor{D}{_c} \tensor{D}{^c} f  \right) \right] \var{\tensor{h}{_a_b}},\\
				\mathrm{II} &= 2\gkappa \var \int_{\spc} \dd[3]{x} \frac{f}{\sqrt{h}} \left( \tensor{h}{_a_c} \tensor{h}{_b_d} \tensor{\pi}{^a^b} \tensor{\pi}{^c^d} - \frac{1}{2} \tensor{h}{_a_b} \tensor{h}{_c_d} \tensor{\pi}{^a^b} \tensor{\pi}{^c^d} \right)\\
				&= 2\gkappa \int_{\spc} \dd[3]{x} \frac{f}{\sqrt{h}} \left( -\frac{1}{2} \tensor{h}{^a^b} \var{\tensor{h}{_a_b}} \right) \left( \tensor{\pi}{_c_d} \tensor{\pi}{^c^d} - \frac{1}{2} \pi^2 \right)\\
				& \qquad \phantom{1} + \frac{f}{\sqrt{h}} \left[ \left( 2 \tensor{h}{_a_c} \var \tensor{h}{_b_d} - \tensor{h}{_c_d} \var \tensor{h}{_a_b} \right) \tensor{\pi}{^a^b} \tensor{\pi}{^c^d} + \left( \tensor{h}{_a_c} \tensor{h}{_b_d} - \frac{1}{2} \tensor{h}{_a_b} \tensor{h}{_c_d} \right) 2 \tensor{\pi}{^c^d} \var \tensor{\pi}{^a^b} \right]\\
				&= 2\gkappa \int_{\spc} \dd[3]{x} \frac{f}{\sqrt{h}} \bigg\{ \left[ -\frac{1}{2} \tensor{h}{^a^b} \left( \tensor{\pi}{_c_d} \tensor{\pi}{^c^d} - \frac{1}{2} \pi^2 \right) + 2 \tensor{\pi}{^c^a} \tensor{\pi}{_c^b} - \pi \tensor{\pi}{^a^b} \right] \var \tensor{h}{_a_b}\\
				& \qquad \phantom{1} + 2 \left( \tensor{\pi}{_a_b} - \frac{1}{2} \pi \tensor{h}{_a_b} \right) \var{\tensor{\pi}{^a^b}} \bigg\},
			\end{align*}
			故
			\begin{equation*}
				\begin{split}
					\fdv{C(f)}{\tensor{h}{_a_b}} &= \frac{\sqrt{h}}{2\gkappa} \left( \tensor{h}{^a^b} \tensor{D}{_c} \tensor{D}{^c} f - \tensor{D}{^b} \tensor{D}{^a} f \right) + f \bigg\{ \frac{\sqrt{h}}{2\gkappa} \left( \tensor{\spacecurR}{^a^b} - \frac{1}{2} \spacecurR \tensor{h}{^a^b} \right)\\
					&\qquad \phantom{1} + \frac{2\gkappa}{\sqrt{h}} \left[ -\frac{1}{2} \tensor{h}{^a^b} \left( \tensor{\pi}{_c_d} \tensor{\pi}{^c^d} - \frac{1}{2} \pi^2 \right) + 2 \tensor{\pi}{^c^a} \tensor{\pi}{_c^b} - \pi \tensor{\pi}{^a^b} \right] \bigg\},\\
					\fdv{C(f)}{\tensor{\pi}{^a^b}} &= \frac{4\gkappa}{\sqrt{h}} f \left( \tensor{\pi}{_a_b} - \frac{1}{2} \pi \tensor{h}{_a_b} \right),
				\end{split}
			\end{equation*}
			于是
			\begin{align*}
				\left\{ C(f), C(f') \right\} &= 2\gkappa \int_{\spc} \dd[3]{x} \left( \fdv{C(f)}{\tensor{h}{_a_b}} \fdv{C(f')}{\tensor{\pi}{^a^b}} - \fdv{C(f')}{\tensor{h}{_a_b}} \fdv{C(f)}{\tensor{\pi}{^a^b}} \right)\\
				&= 4\gkappa \int_{\spc} \dd[3]{x} f' \left( \tensor{h}{^a^b} \tensor{D}{_c} \tensor{D}{^c} f - \tensor{D}{^b} \tensor{D}{^a} f \right) \left( \tensor{\pi}{_a_b} - \frac{1}{2} \pi \tensor{h}{_a_b} \right)\\
				& \qquad \phantom{1} + f f' \left( \tensor{\pi}{_a_b} - \frac{1}{2} \pi \tensor{h}{_a_b} \right) \bigg\{ \left( \tensor{\spacecurR}{^a^b} - \frac{1}{2} \spacecurR \tensor{h}{^a^b}\right)\\ \displaybreak[1]
				&\qquad \phantom{1} + \frac{4\gkappa^2}{h} \left[ -\frac{1}{2} \tensor{h}{^a^b} \left( \tensor{\pi}{_c_d} \tensor{\pi}{^c^d} - \frac{1}{2} \pi^2 \right) + 2 \tensor{\pi}{^c^a} \tensor{\pi}{_c^b} - \pi \tensor{\pi}{^a^b} \right] \bigg\}\\ \displaybreak[1]
				&\qquad \phantom{1} - f \left( \tensor{h}{^a^b} \tensor{D}{_c} \tensor{D}{^c} f' - \tensor{D}{^b} \tensor{D}{^a} f' \right) \left( \tensor{\pi}{_a_b} - \frac{1}{2} \pi \tensor{h}{_a_b} \right)\\ \displaybreak[1]
				& \qquad \phantom{1} - f f' \left( \tensor{\pi}{_a_b} - \frac{1}{2} \pi \tensor{h}{_a_b} \right) \bigg\{ \left( \tensor{\spacecurR}{^a^b} - \frac{1}{2} \spacecurR \tensor{h}{^a^b}\right)\\ \displaybreak[1]
				&\qquad \phantom{1} + \frac{4\gkappa^2}{h} \left[ -\frac{1}{2} \tensor{h}{^a^b} \left( \tensor{\pi}{_c_d} \tensor{\pi}{^c^d} - \frac{1}{2} \pi^2 \right) + 2 \tensor{\pi}{^c^a} \tensor{\pi}{_c^b} - \pi \tensor{\pi}{^a^b} \right] \bigg\}\\ \displaybreak[1]
				&= - 4\gkappa \int_{\spc} \dd[3]{x} \frac{\pi}{2} \left( f' \tensor{D}{_c} \tensor{D}{^c} f - f \tensor{D}{_c} \tensor{D}{^c} f' \right)\\\displaybreak[1]
				&\qquad \phantom{1} + \left( f' \tensor{D}{^b} \tensor{D}{^a} f - f \tensor{D}{^b} \tensor{D}{^a} f' \right) \left( \tensor{\pi}{_a_b} - \frac{1}{2} \pi \tensor{h}{_a_b} \right)\\\displaybreak[1]
				&= - 4\gkappa \int_{\spc} \dd[3]{x} \left( f' \tensor{D}{^b} \tensor{D}{^a} f - f \tensor{D}{^b} \tensor{D}{^a} f' \right) \tensor{\pi}{_a_b} \\
				&= 4\gkappa \int_{\spc} \dd[3]{x} \left( \tensor{D}{^a} f \right) \tensor{D}{^b} \left( f' \tensor{\pi}{_a_b} \right) - \left( \tensor{D}{^a} f' \right) \tensor{D}{^b} \left( f \tensor{\pi}{_a_b} \right)\\
				&= 4\gkappa \int_{\spc} \dd[3]{x} f' \tensor{D}{^a} f \tensor{D}{^b} \tensor{\pi}{_a_b} - f \tensor{D}{^a} f' \tensor{D}{^b} \tensor{\pi}{_a_b}\\
				&= 2\gkappa V(f\tensor{D}{^a}f' - f'\tensor{D}{^a}f).
			\end{align*}
			而矢量约束的变分
			\begin{align*}
				\var{V(v)} &= \var( -2 \int_{\spc} \dd[3]{x} \tensor{v}{_a} \tensor{D}{_b} \tensor{\pi}{^a^b} )\\
				&= \var(2 \int_{\spc} \dd[3]{x} \left( \tensor{D}{_b} \tensor{v}{_a} \right) \tensor{\pi}{^a^b})\\
				&= \var(\int_{\spc} \dd[3]{x} \tensor{\pi}{^a^b} \Ld{v} \tensor{h}{_a_b})\\
				&= \int_{\spc} \dd[3]{x} \left( \Ld{v} \tensor{h}{_a_b} \right) \var{\tensor{\pi}{^a^b}} - \left( \Ld{v} \tensor{\pi}{^a^b} \right) \var{\tensor{h}{_a_b}} + \Ld{v} \left( \tensor{\pi}{^a^b} \var{\tensor{h}{_a_b}} \right)\\
				&= \int_{\spc} \dd[3]{x} \left( \Ld{v} \tensor{h}{_a_b} \right) \var{\tensor{\pi}{^a^b}} - \left( \Ld{v} \tensor{\pi}{^a^b} \right) \var{\tensor{h}{_a_b}}\\
				&\qquad \phantom{1} + \tensor{v}{^c} \tensor{D}{_c} \left( \tensor{\pi}{^a^b} \var{\tensor{h}{_a_b}} \right) + \tensor{\pi}{^a^b} \var{\tensor{h}{_a_b}} \tensor{D}{_c} \tensor{v}{^c}\\
				&= \int_{\spc} \dd[3]{x} \left( \Ld{v} \tensor{h}{_a_b} \right) \var{\tensor{\pi}{^a^b}} - \left( \Ld{v} \tensor{\pi}{^a^b} \right) \var{\tensor{h}{_a_b}} + \tensor{D}{_c} \left( \tensor{v}{^c} \tensor{\pi}{^a^b} \var{\tensor{h}{_a_b}} \right),
			\end{align*}
			得
			\begin{align*}
				\displaybreak[1]
				\fdv{V(v)}{\tensor{h}{_a_b}} &= - \Ld{v} \tensor{\pi}{^a^b},\\
				\fdv{V(v)}{\tensor{\pi}{^a^b}} &= \Ld{v} \tensor{h}{_a_b},
			\end{align*}
			故
			\begin{align*}
				\left\{ V(u), V(v) \right\} &= 2\gkappa \int_{\spc} \dd[3]{x} \left( \fdv{V(u)}{\tensor{h}{_a_b}} \fdv{V(v)}{\tensor{\pi}{^a^b}} - \fdv{V(v)}{\tensor{h}{_a_b}} \fdv{V(u)}{\tensor{\pi}{^a^b}} \right)\\ \displaybreak[1]
				&= -2\gkappa \int_{\spc} \dd[3]{x} \left( \Ld{u} \tensor{\pi}{^a^b} \right) \Ld{v} \tensor{h}{_a_b} - \left( \Ld{v} \tensor{\pi}{^a^b} \right) \Ld{u} \tensor{h}{_a_b}\\
				&= -2\gkappa \int_{\spc} \dd[3]{x} \Ld{u} \left( \tensor{\pi}{^a^b} \Ld{v} \tensor{h}{_a_b} \right) - \Ld{v} \left( \tensor{\pi}{^a^b} \Ld{u} \tensor{h}{_a_b} \right)\\ \displaybreak[1]
				&\qquad \phantom{1} - \tensor{\pi}{^a^b} \Ld{u} \Ld{v} \tensor{h}{_a_b} + \tensor{\pi}{^a^b} \Ld{v} \Ld{u} \tensor{h}{_a_b}\\ \displaybreak[1]
				&= 2\gkappa \int_{\spc} \dd[3]{x} \tensor{\pi}{^a^b} \Ld{[u,v]} \tensor{h}{_a_b}\\ \displaybreak[1]
				&= 2\gkappa V([u,v]),
			\end{align*}
			最后,
			\begin{align*}
				\left\{ V(v), C(f) \right\} &= 2\gkappa \int_{\spc} \dd[3]{x} \left( \fdv{V(v)}{\tensor{h}{_a_b}} \fdv{C(f)}{\tensor{\pi}{^a^b}} - \fdv{C(f)}{\tensor{h}{_a_b}} \fdv{V(v)}{\tensor{\pi}{^a^b}} \right)\\
				&= 2\gkappa \int_{\spc} \dd[3]{x} \left( -\Ld{v} \tensor{\pi}{^a^b} \right) \frac{4\gkappa}{\sqrt{h}} f \left( \tensor{\pi}{_a_b} - \frac{1}{2} \pi \tensor{h}{_a_b} \right)\\
				&\qquad \phantom{1} - \left( \Ld{v} \tensor{h}{_a_b} \right) \frac{\sqrt{h}}{2\gkappa} \left( \tensor{h}{^a^b} \tensor{D}{_c} \tensor{D}{^c} f - \tensor{D}{^b} \tensor{D}{^a} f \right)\\
				&\qquad \phantom{1} - \left( \Ld{v} \tensor{h}{_a_b} \right) f \bigg\{ \frac{\sqrt{h}}{2\gkappa} \left( \tensor{\spacecurR}{^a^b} - \frac{1}{2} \spacecurR \tensor{h}{^a^b} \right)\\ \displaybreak[1]
				&\qquad\quad \phantom{1} + \frac{2\gkappa}{\sqrt{h}} \left[ -\frac{1}{2} \tensor{h}{^a^b} \left( \tensor{\pi}{_c_d} \tensor{\pi}{^c^d} - \frac{1}{2} \pi^2 \right) + 2 \tensor{\pi}{^c^a} \tensor{\pi}{_c^b} - \pi \tensor{\pi}{^a^b} \right] \bigg\}\\ \displaybreak[1]
				&= -2\gkappa \int_{\spc} \dd[3]{x} \frac{4\gkappa}{\sqrt{h}} \left( \tensor{v}{^c} \tensor{D}{_c} \tensor{\pi}{^a^b} - \tensor{\pi}{^a^c} \tensor{D}{_c} \tensor{v}{^b} - \tensor{\pi}{^b^c} \tensor{D}{_c} \tensor{v}{^a} + \tensor{\pi}{^a^b} \tensor{D}{_c} \tensor{v}{^c} \right)\\
				&\qquad \phantom{1} \times \left[ f \left( \tensor{\pi}{_a_b} - \frac{1}{2} \pi \tensor{h}{_a_b} \right) \right]\\ \displaybreak[1]
				&\qquad \phantom{1} + \frac{\sqrt{h}}{\gkappa} \left( \tensor{h}{^a^b} \tensor{D}{_c} \tensor{D}{^c} f - \tensor{D}{^b} \tensor{D}{^a} f \right) \tensor{D}{_a} \tensor{v}{_b}\\ \displaybreak[1]
				&\qquad \phantom{1} + 2 \left( \tensor{D}{_a} \tensor{v}{_b} \right) f \bigg\{ \frac{\sqrt{h}}{2\gkappa} \left( \tensor{\spacecurR}{^a^b} - \frac{1}{2} \spacecurR \tensor{h}{^a^b} \right)\\ \displaybreak[1]
				&\qquad\quad \phantom{1} + \frac{2\gkappa}{\sqrt{h}} \left[ -\frac{1}{2} \tensor{h}{^a^b} \left( \tensor{\pi}{_c_d} \tensor{\pi}{^c^d} - \frac{1}{2} \pi^2 \right) + 2 \tensor{\pi}{^c^a} \tensor{\pi}{_c^b} - \pi \tensor{\pi}{^a^b} \right] \bigg\}\\
				&= -2\gkappa \int_{\spc} \dd[3]{x} \frac{4\gkappa}{\sqrt{h}} f \left( \tensor{\pi}{_a_b} - \frac{1}{2} \pi \tensor{h}{_a_b} \right) \tensor{v}{^c} \tensor{D}{_c} \tensor{\pi}{^a^b}\\
				&\qquad - \frac{8\gkappa}{\sqrt{h}} f \left( \tensor{\pi}{^a^c} \tensor{\pi}{_a^b} - \frac{1}{2} \pi \tensor{\pi}{^b^c} \right) \tensor{D}{_c} \tensor{v}{_b}\\
				&\qquad + \frac{4\gkappa}{\sqrt{h}} f \left( \tensor{\pi}{^a^b} \tensor{\pi}{_a_b} - \frac{1}{2} \pi^2 \right) \tensor{D}{_c} \tensor{v}{^c}\\
				&\qquad + \frac{\sqrt{h}}{\gkappa} {\color{blue} \left[ \left( \tensor{D}{_c} \tensor{D}{^c} f \right) \tensor{D}{_a} \tensor{v}{^a} - \left( \tensor{D}{^b} \tensor{D}{^a} f \right) \tensor{D}{_a} \tensor{v}{_b} \right]}\\
				&\qquad - \frac{\sqrt{h}}{\gkappa}\left( {\color{blue} \tensor{\spacecurR}{^a^b}} - \frac{1}{2} \spacecurR \tensor{h}{^a^b} \right) {\color{blue} \tensor{v}{_b} \tensor{D}{_a} f}\\
				&\qquad + \frac{4\gkappa}{\sqrt{h}} f \bigg[ - \frac{1}{2} \left( \tensor{\pi}{_c_d} \tensor{\pi}{^c^d} - \frac{1}{2} \pi^2 \right) \tensor{D}{_c} \tensor{v}{^c}\\ \displaybreak[1]
				&\qquad\quad + 2 \left( \tensor{\pi}{^c^a} \tensor{\pi}{_c^b} - \frac{1}{2} \pi \tensor{\pi}{^a^b} \right) \tensor{D}{_a} \tensor{v}{_b} \bigg]\\
				&= -2\gkappa \int_{\spc} \dd[3]{x} \frac{4\gkappa}{\sqrt{h}} f \left( \tensor{\pi}{_a_b} - \frac{1}{2} \pi \tensor{h}{_a_b} \right) \tensor{v}{^c} \tensor{D}{_c} \tensor{\pi}{^a^b}\\ \displaybreak[1]
				&\qquad + \frac{2\gkappa}{\sqrt{h}} f \left( \tensor{\pi}{^a^b} \tensor{\pi}{_a_b} - \frac{1}{2} \pi^2 \right) \tensor{D}{_c} \tensor{v}{^c} + \frac{\sqrt{h}}{2\gkappa} \spacecurR v(f)\\
				&= 2\gkappa \int_{\spc} \dd[3]{x} \frac{2\gkappa}{\sqrt{h}} f \left( \tensor{\pi}{_a_b} - \frac{1}{2} \pi \tensor{h}{_a_b} \right) \tensor{\pi}{^a^b} \tensor{D}{_c} \tensor{v}{^c}\\
				&\qquad + \frac{4\gkappa}{\sqrt{h}} \left( \tensor{\pi}{_a_b} - \frac{1}{2} \pi \tensor{h}{_a_b} \right) \tensor{\pi}{^a^b} \tensor{v}{^c} \tensor{D}{_c} f\\
				&\qquad + \frac{4\gkappa}{\sqrt{h}} f \tensor{\pi}{^a^b} \tensor{v}{^c} \tensor{D}{_c} \left( \tensor{\pi}{_a_b} - \frac{1}{2} \pi \tensor{h}{_a_b} \right) - \frac{\sqrt{h}}{2\gkappa} \spacecurR v(f)\\
				&= 2\gkappa C(v(f)) + 2\gkappa \int_{\spc} \dd[3]{x} \frac{2\gkappa}{\sqrt{h}} f \left( \tensor{\pi}{_a_b} - \frac{1}{2} \pi \tensor{h}{_a_b} \right) \tensor{\pi}{^a^b} \tensor{D}{_c} \tensor{v}{^c}\\
				&\qquad {}+ \frac{2\gkappa}{\sqrt{h}} \left( \tensor{\pi}{_a_b} - \frac{1}{2} \pi \tensor{h}{_a_b} \right) \tensor{\pi}{^a^b} v(f)\\
				& \qquad {} + \frac{4\gkappa}{\sqrt{h}} f \tensor{\pi}{^a^b} \tensor{v}{^c} \tensor{D}{_c} \left( \tensor{\pi}{_a_b} - \frac{1}{2} \pi \tensor{h}{_a_b} \right)\\
				&= 2\gkappa C(v(f)) + 2\gkappa \int_{\spc} \dd[3]{x} \frac{2\gkappa}{\sqrt{h}} f \tensor{\pi}{^a^b} \tensor{v}{^c} \tensor{D}{_c} \left( \tensor{\pi}{_a_b} - \frac{1}{2} \pi \tensor{h}{_a_b} \right)\\
				& \qquad {} - \frac{2\gkappa}{\sqrt{h}} f \left( \tensor{\pi}{_a_b} - \frac{1}{2} \pi \tensor{h}{_a_b} \right) \tensor{v}{^c} \tensor{D}{_c} \tensor{\pi}{^a^b}\\
				&= 2\gkappa C(v(f)).
			\end{align*}
			{\normalfont\ttfamily\color{green} 点击返回~\eqref{eq-ADM_constrain_alg}。}
		\end{Proof}

		\begin{Property}[\pageref{eq-tab}页 \eqref{eq-tab}]
			还可讨论标量约束和矢量约束生成的规范变换。任给一个用 $\tensor{h}{_a_b}, \tensor{\pi}{^a^b}$ 构造的张量,例如 $\tensor{t}{_a_b}$,可以算出
			\begin{equation}
				\begin{split}
					\left\{ \tensor{t}{_a_b} , V(v) \right\} &\approx 2\gkappa \Ld{v} \tensor{t}{_a_b},\\
					\left\{ \tensor{t}{_a_b} , C(N) \right\} & \approx 2 \gkappa N \spaceLd{n} \tensor{t}{_a_b}.
				\end{split}
			\end{equation}
		\end{Property}
		\begin{Proof}
			\label{ap-eq-tab}
			以下略去 $t$ 的指标。
			\begin{align*}
				\left\{ t(x), V(v) \right\} &= 2 \gkappa \int_{\spc} \dd[3]{y} \left( \fdv{V(v)}{\tensor{\pi}{^a^b}} \pdv{t}{\tensor{h}{_a_b}} \delta(x-y) - \fdv{V(v)}{\tensor{h}{_a_b}} \pdv{t}{\tensor{\pi}{^a^b}} \delta(x-y) \right)\\
				&=2 \gkappa \fdv{V(v)}{\tensor{\pi}{^a^b}} \pdv{t}{\tensor{h}{_a_b}} - 2 \gkappa \fdv{V(v)}{\tensor{h}{_a_b}} \pdv{t}{\tensor{\pi}{^a^b}},\\
				\left\{ t(x), C(N) \right\} &= 2 \gkappa \int_{\spc} \dd[3]{y} \left( \fdv{C(N)}{\tensor{\pi}{^a^b}} \pdv{t}{\tensor{h}{_a_b}} \delta(x-y) - \fdv{C(N)}{\tensor{h}{_a_b}} \pdv{t}{\tensor{\pi}{^a^b}} \delta(x-y) \right)\\
				&= 2 \gkappa \fdv{C(N)}{\tensor{\pi}{^a^b}} \pdv{t}{\tensor{h}{_a_b}} - 2 \gkappa \fdv{C(N)}{\tensor{h}{_a_b}} \pdv{t}{\tensor{\pi}{^a^b}},
			\end{align*}
			由之前的计算得
			\begin{align*}
				\displaybreak[1]
				\fdv{V(v)}{\tensor{h}{_a_b}} &= - \Ld{v} \tensor{\pi}{^a^b},\\
				\fdv{V(v)}{\tensor{\pi}{^a^b}} &= \Ld{v} \tensor{h}{_a_b},
			\end{align*}
			于是矢量约束生成的规范变换明显为
			\begin{align*}
				\left\{ t, V(v) \right\} &= 2 \gkappa \left( \pdv{t}{\tensor{h}{_a_b}} \Ld{v} \tensor{h}{_a_b} + \pdv{t}{\tensor{\pi}{^a^b}} \Ld{v} \tensor{\pi}{^a^b} \right)\\
				&= 2 \gkappa \Ld{v} t,
			\end{align*}
			对于标量约束,由之前的结果得
			\begin{align*}
				\fdv{C(f)}{\tensor{h}{_a_b}} &= \frac{\sqrt{h}}{2\gkappa} \left( \tensor{h}{^a^b} \tensor{D}{_c} \tensor{D}{^c} f - \tensor{D}{^b} \tensor{D}{^a} f \right) + f \bigg\{ \frac{\sqrt{h}}{2\gkappa} \left( \tensor{\spacecurR}{^a^b} - \frac{1}{2} \spacecurR \tensor{h}{^a^b} \right)\\
				&\qquad \phantom{1} + \frac{2\gkappa}{\sqrt{h}} \left[ -\frac{1}{2} \tensor{h}{^a^b} \left( \tensor{\pi}{_c_d} \tensor{\pi}{^c^d} - \frac{1}{2} \pi^2 \right) + 2 \tensor{\pi}{^c^a} \tensor{\pi}{_c^b} - \pi \tensor{\pi}{^a^b} \right] \bigg\}\\
				\fdv{C(f)}{\tensor{\pi}{^a^b}} &= \frac{4\gkappa}{\sqrt{h}} f \left( \tensor{\pi}{_a_b} - \frac{1}{2} \pi \tensor{h}{_a_b} \right)\\
				&= 2 f \tensor{K}{_a_b}\\
				&= f \spaceLd{n} \tensor{h}{_a_b},
			\end{align*}
			由于 $\fdv{C(f)}{\tensor{h}{_a_b}}$ 包含不与 $f$ 直接成正比的第一项,显然它并不可能直接等于 $-f \Ld{n} \tensor{\pi}{^a^b}$。然而,考虑到运动方程
			\begin{equation*}
				\spaceLd{Nn+\myvec{N}} \tensor{\pi}{^a^b} = - \fdv{H}{\tensor{h}{_a_b}} = - \fdv{\tensor{h}{_a_b}} \left( C(N) + V(\myvec{N}) \right),
			\end{equation*}
			可知 on shell 有
			\begin{equation}
				\fdv{C(f)}{\tensor{h}{_a_b}} \approx -f \spaceLd{n} \tensor{\pi}{^a^b},
			\end{equation}
			于是 on shell 有
			\begin{equation}
				\begin{split}
					\left\{ t, C(N) \right\} &\approx 2 \gkappa \left( \pdv{t}{\tensor{h}{_a_b}} \spaceLd{Nn} \tensor{h}{_a_b} + \pdv{t}{\tensor{\pi}{^a^b}} \spaceLd{Nn} \tensor{\pi}{^a^b} \right)\\
					&\approx 2 \gkappa \spaceLd{Nn} t.
				\end{split}
			\end{equation}
			{\normalfont\ttfamily\color{green} 点击返回~\eqref{eq-tab}。}
		\end{Proof}
