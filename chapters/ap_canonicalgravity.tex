% !TeX root = ../NotesOnLQG.tex

\chapter{第\ref{chp-canonical_gravity}章中的计算}

	\section{ADM formulation}

		\begin{Property}[\pageref{eq-EEq}页\eqref{eq-EEq}]
			\label{pro_EEq}
			易证明,Einstein-Hilbert 作用量
			\begin{equation}
				S_{\text{EH}}[g] = \frac{1}{2\gkappa} \int_M \curR[g] 
			\end{equation}
			的运动方程为真空 Einstein 方程
			\begin{equation}
				Ric - \frac{1}{2} \curR g = 0,
			\end{equation}
			或采取抽象指标形式,写作
			\begin{equation}
				\tensor{\Ric}{_a_b} - \frac{1}{2} \curR \tensor{g}{_a_b} = 0.
			\end{equation}
		\end{Property}

		\begin{Proof}
			\label{prf_EEq}
			考虑时空 $(M,\tensor{g}{_a_b})$ 及 $M$ 上的一族度规 $\tensor{g}{_a_b}(\lambda)$,满足 $\tensor{g}{_a_b}(0)=\tensor{g}{_a_b}$,则任何依赖度规的量 $T$ 的变分为
			\begin{equation}
				\var T\left( \tensor{g}{_a_b} \right) \definedby \left. \dv{T\left( \tensor{g}{_a_b}(\lambda) \right)}{\lambda}  \right|_{\lambda=0}.
			\end{equation}

			注意到
			\begin{equation}
				\var{\Lad} = \underbrace{\sqrt{-g} \tensor{g}{^a^b} \var{\tensor{R}{_a_b}}}_{\RomanNumeralCaps{1}} + \underbrace{\sqrt{-g} \tensor{R}{_a_b} \var{\tensor{g}{^a^b}}}_{\RomanNumeralCaps{2}} + \underbrace{R \var{\sqrt{-g}}}_{\RomanNumeralCaps{3}},\label{eq-varL_EH}
			\end{equation}
			首先考虑与 $\tensor{g}{_a_b}(\lambda)$ 适配的导数算符 $\tensor{\myvaried{\nabla}}{_a}$,假定与 $\Nabla{a}$ 相差 $\tensor{C}{^c_a_b}(\lambda)$,即
			\begin{equation}
				\left( \Nabla{a} - \tensor{\myvaried{\nabla}}{_a} \right) \tensor{\omega}{_b} = \tensor{C}{^c_a_b}(\lambda) \tensor{\omega}{_c},
			\end{equation}
			通过 $\tensor{\myvaried{\nabla}}{_a} \tensor{g}{_b_c}(\lambda) = 0$ ,与计算克氏符类似,可算得
			\begin{equation}
				\tensor{C}{^c_a_b}(\lambda) = \frac{1}{2} \tensor{g}{^c^d}(\lambda) \left( \Nabla{a} \tensor{g}{_b_d}(\lambda) + \Nabla{b} \tensor{g}{_a_d}(\lambda) - \Nabla{d} \tensor{g}{_a_b}(\lambda) \right),\label{eq-Clambda}
			\end{equation}
			进而考虑 $\tensor{\myvaried{\nabla}}{_a}$ 相应的黎曼张量 $\tensor{R}{_a_b_c^d}(\lambda)$,按照定义,有
			\begin{equation}
				\begin{split}
					\tensor{R}{_a_b_c^d}(\lambda) \tensor{\omega}{_d} ={}& 2 \tensor{\myvaried{\nabla}}{_{[a}} \tensor{\myvaried{\nabla}}{_{b]}} \tensor{\omega}{_c}\\
					={}& \tensor{\myvaried{\nabla}}{_a} \left( \Nabla{b} \tensor{\omega}{_c} - \tensor{C}{^d_b_c}(\lambda) \tensor{\omega}{_d} \right) - \tensor{\myvaried{\nabla}}{_b} \left( \Nabla{a} \tensor{\omega}{_c} - \tensor{C}{^d_a_c}(\lambda) \tensor{\omega}{_d} \right)\\
					={}& \left( {\color{red}\Nabla{a} \Nabla{b}\tensor{\omega}{_c}} - {\color{blue}\tensor{C}{^d_a_b}(\lambda) \Nabla{d} \tensor{\omega}{_c}} - \tensor{C}{^d_a_c}(\lambda) \Nabla{b} \tensor{\omega}{_d} \right)\\
					& {}- \left[ \Nabla{a} \left( \tensor{C}{^d_b_c}(\lambda) \tensor{\omega}{_d} \right) - {\color{green}\tensor{C}{^e_a_b}(\lambda) \tensor{C}{^d_e_c}(\lambda) \tensor{\omega}{_d}} - \tensor{C}{^e_a_c}(\lambda) \tensor{C}{^d_b_e}(\lambda) \tensor{\omega}{_d} \right]\\
					& {}- \left( {\color{red}\Nabla{b} \Nabla{a}\tensor{\omega}{_c}} - {\color{blue}\tensor{C}{^d_b_a}(\lambda) \Nabla{d} \tensor{\omega}{_c}} - \tensor{C}{^d_b_c}(\lambda) \Nabla{a} \tensor{\omega}{_d} \right)\\
					& {} + \left[ \Nabla{b} \left( \tensor{C}{^d_a_c}(\lambda) \tensor{\omega}{_d} \right) - {\color{green}\tensor{C}{^e_b_a}(\lambda) \tensor{C}{^d_e_c}(\lambda) \tensor{\omega}{_d}} - \tensor{C}{^e_b_c}(\lambda) \tensor{C}{^d_a_e}(\lambda) \tensor{\omega}{_d} \right]\\
					={} & {\color{red} \tensor{R}{_a_b_c^d} \tensor{\omega}{_d}} - 2 \left( \Nabla{{[a|}} \tensor{C}{^d_{b]}_c}(\lambda) \right) \tensor{\omega}{_d} + 2 \tensor{C}{^e_{[a}_{|c|}}(\lambda) \tensor{C}{^d_{b]}_e}(\lambda) \tensor{\omega}{_d},
				\end{split}
			\end{equation}
			即
			\begin{gather}
				\tensor{R}{_a_b_c^d}(\lambda) = \tensor{R}{_a_b_c^d} - 2 \Nabla{{[a}} \tensor{C}{^d_{b]}_c}(\lambda) + 2 \tensor{C}{^e_c_{[a}}(\lambda) \tensor{C}{^d_{b]}_e}(\lambda),\\
				\tensor{R}{_a_b}(\lambda) = \tensor{R}{_a_c_b^c}(\lambda) = \tensor{R}{_a_b} - 2 \Nabla{{[a}} \tensor{C}{^c_{c]}_b}(\lambda) + 2 \tensor{C}{^e_b_{[a}}(\lambda) \tensor{C}{^c_{c]}_e}(\lambda),
			\end{gather}
			由于 $\tensor{C}{^c_a_b}(0)=0$,求导得
			\begin{equation}
				\var{\tensor{R}{_a_b}} = - 2 \Nabla{{[a}} \var \tensor{C}{^c_{c]}_b} = \Nabla{c} \var \tensor{C}{^c_a_b} - \Nabla{a} \var \tensor{C}{^c_c_b},\label{eq-varR_varC}
			\end{equation}
			而由~\eqref{eq-Clambda}得
			\begin{equation}
				\begin{split}
					\var{\tensor{C}{^c_a_b}} &= \frac{1}{2} \var\tensor{g}{^c^d} \left( \Nabla{a} \tensor{g}{_b_d} + \Nabla{b} \tensor{g}{_a_d} - \Nabla{d} \tensor{g}{_a_b} \right) + \frac{1}{2} \tensor{g}{^c^d} \left( \Nabla{a} \var \tensor{g}{_b_d} + \Nabla{b} \var \tensor{g}{_a_d} - \Nabla{d} \var \tensor{g}{_a_b} \right)\\
					&= \frac{1}{2} \tensor{g}{^c^d} \left( \Nabla{a} \var \tensor{g}{_b_d} + \Nabla{b} \var \tensor{g}{_a_d} - \Nabla{d} \var \tensor{g}{_a_b} \right),\\
					\var{\tensor{C}{^c_c_b}} &= \frac{1}{2} \tensor{g}{^c^d} \left( \Nabla{c} \var \tensor{g}{_b_d} + \Nabla{b} \var \tensor{g}{_c_d} - \Nabla{d} \var \tensor{g}{_c_b} \right)\\
					&= \frac{1}{2} \tensor{g}{^c^d} \Nabla{b} \var \tensor{g}{_c_d},
				\end{split}
			\end{equation}
			代入~\eqref{eq-varR_varC} 得
			\begin{equation}
				\begin{split}
					\var{\tensor{R}{_a_b}} &= \frac{1}{2} \tensor{g}{^c^d} \Nabla{c} \left( \Nabla{a} \var \tensor{g}{_b_d} + \Nabla{b} \var \tensor{g}{_a_d} - \Nabla{d} \var \tensor{g}{_a_b} \right) - \frac{1}{2} \tensor{g}{^c^d} \Nabla{a} \Nabla{b} \var \tensor{g}{_c_d}\\
					&= \frac{1}{2} \tensor{g}{^c^d} \left( \Nabla{c} \Nabla{a} \var \tensor{g}{_b_d} + \Nabla{c} \Nabla{b} \var \tensor{g}{_a_d} - \Nabla{c} \Nabla{d} \var \tensor{g}{_a_b} - \Nabla{a} \Nabla{b} \var \tensor{g}{_c_d} \right),
				\end{split}
			\end{equation}
			则~\eqref{eq-varL_EH} 中的 \RomanNumeralCaps{1} 为
			\begin{equation}
				\begin{split}
					\RomanNumeralCaps{1} &= \sqrt{-g} \tensor{g}{^a^b} \var{\tensor{R}{_a_b}}\\
					&= \frac{1}{2} \sqrt{-g} \left( \tensor{\nabla}{^d} \tensor{\nabla}{^b} \var \tensor{g}{_b_d} + \tensor{\nabla}{^d} \tensor{\nabla}{^a} \var \tensor{g}{_a_d} - \tensor{g}{^a^b} \tensor{\nabla}{^d} \Nabla{d} \var \tensor{g}{_a_b} - \tensor{g}{^c^d} \tensor{\nabla}{^a} \Nabla{a} \var \tensor{g}{_c_d} \right)\\
					&= \sqrt{-g} \left( \tensor{\nabla}{^a} \tensor{\nabla}{^b} \var{\tensor{g}{_a_b}} - \tensor{g}{^b^c} \tensor{\nabla}{^a} \Nabla{a} \var{\tensor{g}{_b_c}} \right)\\
					&= \sqrt{-g} \Nabla{a} \tensor{v}{^a},
				\end{split}
			\end{equation}
			其中
			\begin{equation}
				\tensor{v}{^a} \definedby \tensor{\nabla}{^b} \var{\tensor{g}{_a_b}} - \tensor{g}{^b^c} \Nabla{a} \var{\tensor{g}{_b_c}},
			\end{equation}
			故这一项仅为边界项。

			对 $\tensor{\delta}{^a_b} = \tensor{g}{^a^c} \tensor{g}{_c_b}$ 两边变分得关系
			\begin{equation}
				\begin{split}
					\var{\tensor{g}{_a_b}} = - \tensor{g}{_a_c} \tensor{g}{_b_d} \var{\tensor{g}{^c^d}} \qc \var{\tensor{g}{^a^b}} = - \tensor{g}{^a^c} \tensor{g}{^b^d} \var{\tensor{g}{_c_d}},
				\end{split}
			\end{equation}
			故得~\eqref{eq-varL_EH} 中的 \RomanNumeralCaps{2} 为
			\begin{equation}
				\begin{split}
					\RomanNumeralCaps{2} &= \sqrt{-g} \tensor{R}{_a_b} \var{\tensor{g}{^a^b}}\\
					&= - \sqrt{-g} \tensor{R}{_a_b} \tensor{g}{^a^c} \tensor{g}{^b^d} \var{\tensor{g}{_c_d}}\\
					&= - \sqrt{g} \tensor{R}{^a^b} \var{\tensor{g}{_a_b}},
				\end{split}
			\end{equation}
			最后考虑 \RomanNumeralCaps{3},注意到若记 $\tensor{\nvol}{_a_b_c_d}$ 为坐标体元,有行列式表达式
			\begin{equation}
				g = \frac{1}{4!} \tensor{\nvol}{^a^b^c^d} \tensor{\nvol}{^e^f^g^h} \tensor{g}{_a_e} \tensor{g}{_b_f} \tensor{g}{_c_g} \tensor{g}{_d_h},
			\end{equation}
			则
			\begin{equation}
				\begin{split}
					\var g &= \frac{1}{3!} \tensor{\nvol}{^a^b^c^d} \tensor{\nvol}{^e^f^g^h} \tensor{g}{_a_e} \tensor{g}{_b_f} \tensor{g}{_c_g} \var \tensor{g}{_d_h},
				\end{split}
			\end{equation}
			记
			\begin{equation}
				\tensor{T}{^d^h} \definedby \frac{1}{3!} \tensor{\nvol}{^a^b^c^d} \tensor{\nvol}{^e^f^g^h} \tensor{g}{_a_e} \tensor{g}{_b_f} \tensor{g}{_c_g},
			\end{equation}
			则显然它对称,迹为 $4g$,有
			\begin{equation}
				\tensor{T}{^d^h} = g \tensor{g}{^d^h} + \tensor{S}{^d^h},
			\end{equation}
			其中 $\tensor{S}{^d^h}$ 无迹。另一方面,
			\begin{equation}
				\begin{split}
					\tensor{T}{^d^h} \tensor{g}{_h_l} &= \frac{1}{3!} \tensor{\nvol}{^a^b^c^d} \tensor{\nvol}{^e^f^g^h} \tensor{g}{_a_e} \tensor{g}{_b_f} \tensor{g}{_c_g} \tensor{g}{_h_l}\\
					&= \frac{1}{3!} \tensor{\nvol}{^\mu^\nu^\sigma^\rho} \tensor{\nvol}{^\alpha^\beta^\gamma^\delta} \tensor{g}{_\mu_\alpha} \tensor{g}{_\nu_\beta} \tensor{g}{_\sigma_\gamma} \tensor{g}{_\delta_\eta} \tensor{\left( \pdv{x^\rho} \right)}{^d} \tensor{\left( \dd{x^\eta} \right)}{_l}\\
					&= \tensor{\nvol}{^0^1^2^\rho} \tensor{\nvol}{^\alpha^\beta^\gamma^\delta} \tensor{g}{_0_\alpha} \tensor{g}{_1_\beta} \tensor{g}{_2_\gamma} \tensor{g}{_\delta_\eta} \tensor{\left( \pdv{x^\rho} \right)}{^d} \tensor{\left( \dd{x^\eta} \right)}{_l}\\
					&= \tensor{\nvol}{^\alpha^\beta^\gamma^\delta} \tensor{g}{_0_\alpha} \tensor{g}{_1_\beta} \tensor{g}{_2_\gamma} \tensor{g}{_\delta_3} \tensor{\left( \pdv{x^3} \right)}{^d} \tensor{\left( \dd{x^3} \right)}{_l}\\
					&= g\tensor{\delta}{^d_l},
				\end{split}
			\end{equation}
			其中倒数第二行中 $\eta$ 必取 $3$ 是因为,否则,不妨设 $\eta$ 取 $0$,则 $\alpha$ 与 $\delta$ 对称,与 $\tensor{\nvol}{^\alpha^\beta^\gamma^\delta}$ 缩并为 $0$。则可知 $\tensor{S}{^d^h}=0$,
			\begin{equation}
				\tensor{T}{^d^h} = \frac{1}{3!} \tensor{\nvol}{^a^b^c^d} \tensor{\nvol}{^e^f^g^h} \tensor{g}{_a_e} \tensor{g}{_b_f} \tensor{g}{_c_g} = g \tensor{g}{^d^h},
			\end{equation}
			故
			\begin{equation}
				\var{g} = g \tensor{g}{^a^b} \var{\tensor{g}{_a_b}},
			\end{equation}
			于是
			\begin{equation}
				\begin{split}
					\RomanNumeralCaps{3} &= R \var{\sqrt{-g}}\\
					&= - \frac{R}{2\sqrt{-g}} \var{g}\\
					&= \frac{1}{2} \sqrt{-g} R \tensor{g}{^a^b} \var{\tensor{g}{_a_b}},
				\end{split}
			\end{equation}
			故
			\begin{equation}
				\var{\Lad} = \RomanNumeralCaps{1} + \RomanNumeralCaps{2} + \RomanNumeralCaps{3} = \sqrt{-g} \Nabla{a} \tensor{v}{^a} - \sqrt{-g} \left( \tensor{R}{^a^b} - \frac{1}{2} R \tensor{g}{^a^b} \right) \var{\tensor{g}{_a_b}},
			\end{equation}
			可得真空场方程。{\normalfont\ttfamily\color{green} 点击返回~\eqref{eq-EEq}。}
		\end{Proof}

		\begin{Remark}
			\begin{enumerate}
				\item 也可采用矩阵的语言,使用伴随矩阵为工具计算 $\delta g$,参见\inlinecite{liang3}。
				\item 也可考察 $\La = R \form{\vol}$,则需要计算适配体元的变分。梁灿彬老师在 \inlinecite{liang3} 的下册第~9~页中写道
				\begin{quote}
					……对 $\tensor{g}{_a_b}$ 变分时就必须考虑 $\form{\vol}$ 的相应变分,从而给计算带来麻烦。
				\end{quote}
				进而得出用标量密度 $\Lad$ 更合适的结论。然而鄙人实际算了发现没觉得体元的变分有多复杂……见下文。

				记 $\form{\vol}(\lambda)$ 是与 $\tensor{g}{_a_b}(\lambda)$ 适配的体元,则
				\begin{equation}
					\tensor{g}{^a^e}(\lambda) \tensor{g}{^b^f}(\lambda) \tensor{g}{^c^g}(\lambda) \tensor{g}{^d^h}(\lambda) \tensor{\vol}{_a_b_c_d}(\lambda) \tensor{\vol}{_e_f_g_h}(\lambda) = - 4!,
				\end{equation}
				对 $\lambda$ 求导得
				\begin{equation}
					\begin{split}
						0 &= 4 \left( \var \tensor{g}{^a^e} \right) \tensor{g}{^b^f} \tensor{g}{^c^g} \tensor{g}{^d^h} \tensor{\vol}{_a_b_c_d} \tensor{\vol}{_e_f_g_h} + 2 \tensor{g}{^a^e} \tensor{g}{^b^f} \tensor{g}{^c^g} \tensor{g}{^d^h} \tensor{\vol}{_a_b_c_d} \var \tensor{\vol}{_e_f_g_h}\\
						&= - 4 \times 3! \times \tensor{g}{_a_e} \var{\tensor{g}{^a^e}} + 2 \tensor{\vol}{^e^f^g^h} \var{\tensor{\vol}{_e_f_g_h}},
					\end{split}
				\end{equation}
				故
				\begin{equation}
					\begin{split}
						\tensor{\vol}{^a^b^c^d} \var{\tensor{\vol}{_a_b_c_d}} &= - 2 \times 3! \times \tensor{g}{^a^b} \var{\tensor{g}{_a_b}}\\
						&= - \frac{1}{2} 4! \tensor{g}{^a^b} \var{\tensor{g}{_a_b}}\\
						&= \frac{1}{2} \tensor{\vol}{^e^f^g^h} \left( \tensor{\vol}{_e_f_g_h} \tensor{g}{^a^b} \var{\tensor{g}{_a_b}} \right),
					\end{split}
				\end{equation}
				两边取对偶形式去掉 $\form{\vol}$,知
				\begin{equation}
					\var{\tensor{\vol}{_a_b_c_d}} = \frac{1}{2} \tensor{\vol}{_a_b_c_d} \tensor{g}{^e^f} \var{\tensor{g}{_e_f}}.
				\end{equation}

				算完适配体元变分后,即可算得
				\begin{equation}
					\begin{split}
						\var{\La} &= \tensor{g}{^a^b} \var(\tensor{R}{_a_b}) \form{\vol} - \tensor{R}{^a^b} \var(\tensor{g}{_a_b}) \form{\vol} + R \var{\form{\vol}}\\
						&= \left( \Nabla{a} \tensor{v}{^a} \right) \form{\vol} - \left( \tensor{R}{^a^b} - \frac{1}{2} R \tensor{g}{^a^b} \right) \var{\tensor{g}{_a_b}} \form{\vol},
					\end{split}
				\end{equation}
				从而得到场方程。{\normalfont\ttfamily\color{green} 点击返回~\eqref{eq-EEq}。}
			\end{enumerate}

			\begin{Property}[\pageref{pro-KLnh} 页命题~\ref{pro-KLnh}]
				\begin{equation}
					\tensor{K}{_a_b} = \frac{1}{2} \Ld{n} \tensor{h}{_a_b},
				\end{equation}
				其中 $\Ld{n}$ 表示沿 $\tensor{n}{^a}$ 的李导数。
			\end{Property}
			\begin{Proof}
				\label{prf-KLnh}
				由李导数公式
				\begin{equation}
					\Ld{v} \tensor{T}{^{a_1 \cdots a_k}_{b_1 \cdots b_l}} = \tensor{v}{^c} \Nabla{c} \tensor{T}{^{a_1 \cdots a_k}_{b_1 \cdots b_l}} - \sum_{i=1}^k \tensor{T}{^{a_1 \cdots c \cdots a_k}_{b_1 \cdots b_l}} \Nabla{c} \tensor{v}{^{a_i}} + \sum_{j=1}^l \tensor{T}{^{a_1 \cdots a_k}_{b_1 \cdots c \cdots b_l}} \Nabla{{b_j}} \tensor{v}{^c}
				\end{equation}
				知
				\begin{equation}
					\Ld{v} \tensor{g}{_a_b} = 2 \Nabla{{(a}} \tensor{v}{_{b)}}
				\end{equation}
				及
				\begin{equation}
					\Ld{n} \tensor{n}{_a} = \tensor{n}{^b} \Nabla{b} \tensor{n}{_a} + \tensor{n}{_b} \Nabla{a} \tensor{n}{^b} = \tensor{n}{^b} \Nabla{b} \tensor{n}{_a},
				\end{equation}
				故
				\begin{equation}
					\begin{split}
						\Ld{n} \tensor{h}{_a_b} &= \Ld{n} \left( \tensor{g}{_a_b} + \tensor{n}{_a} \tensor{n}{_b} \right)\\
						&= 2 \Nabla{{(a}} \tensor{n}{_{b)}} + 2 \tensor{n}{_{(a}} \Ld{n} \tensor{n}{_{b)}}\\
						&= 2 \Nabla{{(a}} \tensor{n}{_{b)}} + 2 \tensor{n}{_{(a}} \tensor{n}{^c} \Nabla{{|c|}} \tensor{n}{_{b)}}\\
						&= 2 \tensor{\delta}{^c_{(a}} \Nabla{{|c|}} \tensor{n}{_b} + 2 \tensor{n}{_{(a}} \tensor{n}{^c} \Nabla{{|c|}} \tensor{n}{_{b)}}\\
						&= 2 \tensor{K}{_{(ab)}}\\
						&= 2 \tensor{K}{_a_b},
					\end{split}
				\end{equation}
				证毕。{\normalfont\ttfamily\color{green} 点击返回命题~\ref{pro-KLnh}。}
			\end{Proof}
		\end{Remark}
