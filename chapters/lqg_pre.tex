% !TeX root = ../NotesOnLQG.tex

\chapter{正则圈量子引力:量子化程序}

	\label{chp-Quantization}
	\section{量子化的准备工作}

		\label{sec-pre}
		在量子化之前,还需作一些讨论。

		\paragraph{规范群的选择} 如前文所述,纯引力的情况下,Ashtekar 联络 $\tensor{A}{^i_a}$ 是 $\so{3}$ 值联络,理论具有局域 $\SO{3}$ 规范对称性。但为了方便量子化,也为了考虑与旋量场的耦合——即引入费米子,从这里起我们将规范群替换为 $\SO{3}$ 的双覆盖群 $\SU{2}$,$\tensor{A}{^i_a}$ 理解为 $\su{2}$ 值联络。取 $W$ 到 $\su{2}$ 的同构 $\tensor{\tau}{_i}$,使得 $\tensor{\tau}{_\alpha} \definedby \tensor{\tau}{_i} \tensor{\xi}{^i_\alpha}, \alpha=1,2,3$ 在 Killing 度规下正交归一,例如$\frac{\ii}{2}$倍的三个泡利矩阵。则 $\tensor{A}{_a} = \tensor{A}{^i_a} \tensor{\tau}{_i}$ 是 $\su{2}$ 值一形式。

		\nomenclature{$\tensor{\tau}{_i} = \frac{\ii}{2} \tensor{\sigma}{_i}$}{$\su{2}$的一组正交归一基底}

		\paragraph{约束代数的分析} 注意到约束构成的泊松代数~\eqref{eq-constrain_algebra} 中,右边是依赖变量 $\tensor{A}{^i_a}$ 和 $\tensor{\tilde{P}}{^a_i}$ 的,故约束代数并非李代数,这给量子化带来一定的麻烦。注意到所有高斯约束 $\left\{ G(\Lambda) \right\}$ 不仅是泊松代数的子代数,还是双边理想,因此可以考虑先独立解出高斯约束,然后考虑约化约束代数
		\begin{equation}
			\begin{split}
				\left\{ C_{\mathrm{Diff}}(u), C_{\text{Diff}}(v) \right\} &= C_{\text{Diff}}([u,v]),\\
				\left\{ C_{\text{Diff}}(v), C_{\text{H}}(f) \right\} &= - C_{\text{H}}(v(f)),\\
				\left\{ C_{\text{H}}(f), C_{\text{H}}(f') \right\} &= - C_{\text{Diff}}(S(f,f')),\label{eq-reduced_constrain_algebra}
			\end{split}
		\end{equation}
		进一步解约束则比较麻烦,因为微分同胚约束和哈密顿约束不能分开处理。这给哈密顿算符的构造带来一定的模糊性\cite{Han2005,Thiemann1996aw}。2006年,Thomas Thiemann 发表了“The Phoenix Project”\cite{Thiemann2003zv},将约束代数~\eqref{eq-reduced_constrain_algebra} 重写,引入了 Master Constraint Programme,后文会回到这个问题上。

	\section{第一类约束系统的量子化}

		\label{sec-Quantization}
		之前在第\ref{chp-canonical_gravity}章第\ref{section-QGD}节中介绍了 Dirac 量子化的简要思想,但由于原本的 Dirac 量子化方法存在一些不足,尤其是在物理状态上定义的 Hilbert 空间结构需要额外附加,会导致一些问题。在解决其问题上取得进展的方法有几何量子化,coherent states 路径积分,$C^*$ 代数方法,Algebraic Quantization,Refined Algebraic Quantization等。这里参考 \cite{Han2005,Thiemann0210094, Thiemann2007,arXiv9812024} 对 Refined Algebraic Quantization 进行简要概括
		%,并略去一些虽然数学上很重要,但对物理无关紧要的细节(例如对物理量的泊松代数取泛包络代数)
		。
		
		\subsection{Refined Algebraic Quantization(RAQ)}
		\begin{enumerate}
			\item \emph{相空间与约束}

					量子化程序的出发点是给定相空间 $\left( \phasespace{M}, \{\cdot, \cdot\} \right)$ ,哈密顿量 $H$ 及一些第一类约束 $\left\{ C_I \right\}$。

			\item \emph{选择极化}
			
					相空间的极化是指从相空间中选定位型变量 ${q}$,用数学语言来讲,是选择 $\phasespace{M}$ 的一个拉格朗日子流形 $\configurationspace{C}$。物理上感兴趣的情况下,往往有 $\phasespace{M} \cong \CTB{\configurationspace{Q}}$,则取 $\configurationspace{C} = \configurationspace{Q}$ 即可。

			\item \emph{物理量的$*$-代数}
			
					这一步要选择基本物理量构成的李$*$-代数 $\staralgebra{B}$。对量子力学的情况,可选用由 $\left\{ {q}, {p} \right\}$ 生成的代数,它具有泊松括号 $ \left\{ q^i, p_j \right\} = \tensor{\delta}{^i_j}$,但在讨论无穷自由度的场论时右边会变成 $\delta$ 函数,不便使用。于是场论情况可用光滑函数 $f \in C^\infty(\configurationspace{C})$ 代替 $q$,动量的哈密顿矢量场 $v_p(f) \definedby \left\{ p, f \right\}$ 代替 $p$,并定义一个新的括号 $\left[ (f,v_p), \left( f',v_{p'} \right) \right] \definedby \left( v(f') - v'(f), \left[ v,v' \right] \right)$,则由所有 $(f,v_p)$ 形成的李代数记为 $\staralgebra{B}$,复共轭是其上的一个 $*$ 运算。$\staralgebra{B}$ 的泛包络代数记为 $\staralgebra{A}$,这是一个泊松 $*$-代数,称为物理量的量子代数。

			\item \emph{$*$-代数的不可约表示}

					选择量子代数 $\staralgebra{A}$ 的不可约表示。但为了方便,这里忽略掉数学细节,而直接考虑 $\staralgebra{B}$ 的表示 $\pi \colon \staralgebra{B} \rightarrow \Linear{\Hil_{\text{kin}}}$,使得 $\forall a,b \in \staralgebra{B}$ 有
					\begin{equation}
						\begin{split}
							{\pi(a)}^\dagger &= \pi(a^*),\\
							\left[ \pi(a), \pi(b) \right] &= \ii \hbar \pi\left( \left[ a,b \right] \right),\label{eq-representation}
						\end{split}
					\end{equation}
					实际上式~\eqref{eq-representation} 中的等号有定义域的问题,而且有 Groenewald-van Hove 定理的问题,为了避免这些数学的麻烦,可修改 $\staralgebra{B}$ ,用 $C^\infty(\configurationspace{C})$ 中有紧支集的函数所形成的子代数代替 $C^\infty(\configurationspace{C})$。

					通常可以选择 $\Hil_{\text{kin}}$ 为 $L^2(\bar{\configurationspace{C}},\mu)$,其中 $\bar{\configurationspace{C}}$ 是 $\configurationspace{C}$ 的 distributional extension, $\mu$ 是 $\extentedconfigurationspace{C}$ 上的积分测度。例如,在闵氏时空上的标量场的情况下,$\configurationspace{C}$ 是 $\mathbb{R}^{3}$ 上的速降函数集合,故 $\extentedconfigurationspace{C}$ 是 $\mathbb{R}^{3}$ 上的缓增分布集合,$\mu$ 是 $\extentedconfigurationspace{C}$ 上的归一化高斯分布。

			\item \emph{约束算符和哈密顿算符}
			
					约束和哈密顿函数往往不在 $\staralgebra{B}$ 中,除非它们只含动量的至多一次幂项,否则,$\hat{C}_I \definedby \pi(C_I)$ 和 $\hat{H} \definedby \pi(H)$ 这样的表达式会遇到 ordering ambiguity,在场论情形还可能需要正规化。我们假定 $\hat{H}$ 是对称算子(symmetric operator)\footnote{即,任取 $\psi,\phi$ 属于 $\hat{H}$ 定义域,有 $\ip*{\hat{H}\psi}{\phi} = \ip*{\psi}{\hat{H} \phi}$。},且 $\hat{C}_I$ 都是至少可闭的(closable)\footnote{算符 $A$ 是可闭的意为 $A$ 和 $A^\dagger$ 都是稠定的。},则一般可以找到一个子集 $\Dkin \subset \Hil_{\text{kin}}$,使得所有约束算符 $\hat{C}_I$ 及其对偶算符 $\hat{C}_I^\dagger$ 和哈密顿算符 $\hat{H}$ 都在 $\Dkin$ 上稠定,且是这些算符的不变子空间,即 $\hat{H} \Dkin \subset \Dkin$ 且任取 $I$ 有 $\hat{C}_I \Dkin \subset \Dkin$。

			\item \emph{施加约束}
			
					经典理论中求解约束方程 $C_I=0$ 并约化相轨道在量子理论中变为求解约束算符方程 
					\begin{equation}
						\hat{C}_I l =0,
					\end{equation}
					则 $l$ 求解了约束并且是规范不变的态,因为约束算符是量子规范变换的生成元。

					但由于连续取值的经典量 $C_I$ 在量子化之后得到的算符 $\hat{C}_I$ 可能具有离散点谱或连续谱,当且仅当 $0$ 属于所有 $\hat{C}_I$ 的点谱时,上述方程 $\hat{C}_I l =0$ 才可能对所有 $I$ 有共同解。否则,就要考虑在更大的分布空间中求解 $l$。这里选择为 $\Dkin$ 的代数对偶空间 $\Dkin^*$ ,于是得到 \emph{Gel'fand triple} 或称 \emph{rigged Hilbert space}
					\begin{equation}
						\Dkin \hookrightarrow \Hil_{\text{kin}} \hookrightarrow \Dkin^*,
					\end{equation}
					量子约束方程变为在 $\Dkin^*$ 中求解一个子空间 $\Dphys^*$ 使得其中的任何元素 $l$ 满足
					\begin{equation}
						\left( \hat{C}_I^\prime l \right)(f) \definedby l \left( \hat{C}_I^\dagger f \right) = 0 \qc \forall f \in \Dkin, \forall I,
					\end{equation}
					其中第一个等号左边定义了约束在 $\Dkin^*$ 上的一个反线性对偶表示。

			\item \emph{量子反常}
			
					经典层面第一类约束系统即存在一族结构函数 $\tensor{f}{_I_J^K}$ ,使得
					\begin{equation}
						\left\{ C_I, C_J \right\} = \tensor{f}{_I_J^K} C_K,
					\end{equation}
					但在量子理论中,由于结构函数也成为算符,$\tensor{\hat{f}}{_I_J^K}$ 和 $\hat{C}_K$ 可能有 ordering ambiguity。例如,可能有
					\begin{equation}
						\comm{\hat{C}_I}{\hat{C}_J} = \ii\hbar \hat{C}_K \tensor{\hat{f}}{_I_J^K} = \ii\hbar \left( \comm{\hat{C}_K}{\tensor{\hat{f}}{_I_J^K}} + \tensor{\hat{f}}{_I_J^K} \hat{C}_K \right),
					\end{equation}
					则若 $\comm{\hat{C}_K}{\tensor{\hat{f}}{_I_J^K}}$ 不是约束算符,它便成为了新的约束,使得量子理论比相对应的经典理论具有更少的物理自由度,这种情况称为 \emph{反常}(anomaly),是理论应当避免的。

			\item \emph{狄拉克观测量与物理内积}
			
					如前所述,除非 $0$ 属于所有约束算符的点谱,否则 $\Hkin \cap \Dphys^* = \left\{ 0 \right\}$ ,于是 $\Dphys^*$ 上需要另外定义内积,而不能继承 $\Hkin$ 上的 $\ip{\cdot}_{\text{kin}}$。$\ip{\cdot}_{\text{phys}}$ 应使得物理量是自伴算子,为此定义\emph{狄拉克观测量}(Dirac observable)。 
					\begin{Definition}
						一个\emph{强}狄拉克观测量(strong Dirac Observable)是指 $\Hkin$ 上的一个算符 $\hat{O}$ ,满足 $\hat{O}$ 和 $\hat{O}^\dagger$ 都在 $\Dkin$ 上稠定,并且任取 $I$ 有 $\comm{\hat{O}}{\hat{C}_I} = 0$;一个\emph{弱}狄拉克观测量(weak Dirac observable)是指 $\Hkin$ 上的一个算符 $\hat{O}$ ,满足 $\hat{O}^\prime \Dphys^* \subset \Dphys^*$。
					\end{Definition}
					强狄拉克观测量是满足规范不变条件 $\left\{ O, C_I \right\} = 0, \forall I$ 的经典物理量 $O$ 的量子版本,而弱狄拉克观测量则是较弱的条件 $\left. \left\{ O, C_I \right\} \right|_{{\phasespace{M}^\prime}} =0$ 的量子版本,其中 $\phasespace{M}^\prime$ 是相空间 $\phasespace{M}$ 中由 $C_I = 0, \forall I$ 决定的约束曲面。 

					通过适当选取 $\Dphys^*$ 的子集 $\Hphys$ 及拓扑,可以得到另一个 Gel'fand triple
					\begin{equation}
						\Dphys \hookrightarrow \Hphys \hookrightarrow \Dphys^*,
					\end{equation}
					使得所有强狄拉克观测量 $\hat{O}$ 可以看作在 $\Dphys$ 上稠定的算子。在 $\Hphys$ 上选择物理内积 $\ip{\cdot	}_{\text{phys}}$,使 $\hat{O}^\prime$ 都是自伴的。

			\item \emph{经典极限}
			
					最后,需要检查理论的经典极限。 $\Hphys$ 需要包括半经典态,但这取决于具体的选择。记 $m\in \phasespace{M}$ 所在的规范等价类为 $[m]$,设有一组 由 $[m]$ 标记的态 $\psi_{[m]}$,粗略来说,若满足对经典狄拉克观测量 $O$ 及其对应算符 $\hat{O}'$ 有
					\begin{gather}
						\lim_{\hbar \rightarrow 0} \abs{\frac{\ev**{\hat{O}'}{\psi_{[m]}}_{\text{phys}}}{O(m)} - 1} = 0,\\
						\lim_{\hbar \rightarrow 0} \abs{\frac{\ev**{\left( \hat{O}^\prime \right)^2}{\psi_{[m]}}_{\text{phys}}}{\ev**{\hat{O}^\prime}{\psi_{[m]}}_{\text{phys}}^2} - 1} = 0,
					\end{gather}
					则 $\psi_{[m]}$ 构成一组半经典态。这样才能确信量子理论的构造有正确的经典极限。
		\end{enumerate}

	\subsection{Master Constraint Approach}
		注意到在 Refined Algebraic Quantization 中并没有提供找出 $\Hphys$ 和 $\ip{\cdot}_{\text{phys}}$ 的可以一步步进行的“算法”;另一方面,沿着 RAQ 进行量子化在求解量子哈密顿约束时仍很复杂。Thiemann 为求解量子约束提出了 \emph{Master Constraint Approach},对于求解广义相对论的约束似乎非常合适\cite{Thiemann2003zv}。这里予以简单介绍。

		设任选一实值正定的“矩阵” $K_{IJ}$\footnote{在场论情形 $I,J$ 包含连续指标,如坐标 $x$,故矩阵一词加引号。此时,$K$ 还需对 $x$ 满足适当的速降条件。},定义\emph{主约束}(master constraint)
		\begin{equation}
			\masterconstraint{M} \definedby \frac{1}{2} \sum_{I,J} \bar{C}_I K_{IJ} {C}_J,
		\end{equation}
		则经典理论中 $\masterconstraint{M}=0$ 当且仅当所有约束为零。并且,可以验证 $O$ 为弱狄拉克观测量当且仅当 $\left. \left\{ \left\{ O, \masterconstraint{M} \right\}, O \right\} \right|_{\masterconstraint{M} = 0} =0$。对于量子理论,容易想到在 $\Hkin$ 上考察\emph{主约束方程}(Master Equation)
		\begin{equation}
			\hat{\masterconstraint{M}} \psi = 0,
		\end{equation}
		很容易明确此式的意义。由于 $\masterconstraint{M}$ 经典层面正定,合适的选择可以使 $\hat{\masterconstraint{M}}$ 是自伴的,于是根据谱定理,$\Hkin$ 有direct integral分解
		\begin{equation}
			\begin{gathered}
				\Hkin \cong \int_{\sigma(\masterconstraint{M})}^{\oplus} \dd{\mu}(\lambda) \Hkin^\oplus(\lambda),\\
				\ip{\psi}{\phi}_{\text{kin}} = \int_{\sigma(\masterconstraint{M})} \dd{\mu}(\lambda) \ip{\psi(\lambda)}{\phi(\lambda)}_{\Hkin^\oplus(\lambda)},\label{eq-direct_integral}
			\end{gathered}
		\end{equation}
		其中第一式的 $\cong$ 表示酉等价,$\sigma(\hat{\masterconstraint{M}})$ 表示 $\hat{\masterconstraint{M}}$ 的谱,而 $\mu$ 是谱测度。注意到 $0$ 不一定是 $\hat{\masterconstraint{M}}$ 的谱点,例如若某系统有$\left\{ x,y,p,q \right\}$ 四个正则坐标,$\masterconstraint{M} = \frac{1}{2} \left( \omega^2 x^2+y^2 \right)$,经典理论中这意味着将 $x,y$ 从正则坐标中删除,而量子主约束算符多出零点能 $\frac{1}{2} \hbar \omega$。故应重定义主约束算符 $\hat{\masterconstraint{M}} \mapsto \hat{\masterconstraint{M}} - \lambda_0^{\masterconstraint{M}} \hat{\II}_{\Hkin}$,其中 $\lambda_0^{\masterconstraint{M}} \definedby \inf{\sigma(\hat{\masterconstraint{M}})}$。重定义后,取 $\Hphys \definedby \Hkin^\oplus(0)$,其上自带内积 $\ip{\cdot}_{\Hkin^{\oplus}(0)}$ 可作为物理内积。