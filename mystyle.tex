%\usepackage[scheme=plain]{ctex}
%\usepackage{amsmath}
%\usepackage{amssymb}
%\usepackage{amsthm}
\usepackage{framed}
\usepackage[amsmath,hyperref,thmmarks,framed]{ntheorem}
%\usepackage{mathrsfs} %定义了类似铜版体的\mathscr
%\usepackage{upgreek}
%\usepackage{textcomp}
\usepackage{physics}
\usepackage{cancel}
\usepackage{geometry}
%\usepackage{tabularx}
%\usepackage{hyperref}
%\usepackage{bookmark}
\usepackage{xcolor}
%\usepackage{graphicx}
%\usepackage{enumitem}
%\usepackage{fancyvrb}%能用颜文字卖萌了,2333
%\usepackage{siunitx}%生成标准格式的国际单位
\usepackage{tensor}
\usepackage{tikz}
%\usepackage{listings}
%\usepackage{fontspec}
%\usepackage{minted}
%\usepackage{pifont}
\usepackage{bm}
\usepackage{nomencl}

% \setmathfont{latinmodern-math.otf}

\usetikzlibrary{arrows.meta}

\newcommand{\myarrow}{-{Latex[length=5pt 6,width'=0pt 0.3]}}

\hypersetup{
	colorlinks=true,
	linkcolor=blue,
	pdftitle={Notes on Loop Quantum Gravity},
	pdfauthor={Yingjie Wang}
}

{
	\theoremstyle{plain}
	\theoremsymbol{\ensuremath{\clubsuit}}
	\theoremseparator{.}
	\theoremprework{\bigskip\hrule\smallskip}
	\theorempostwork{\smallskip\hrule\bigskip}
	\newtheorem{Definition}{定义}
}
{
	\theoremstyle{plain}
	\theoremheaderfont{\normalfont\bfseries}
	\theorembodyfont{\slshape}
	\theoremsymbol{\ensuremath{\diamondsuit}}
	\theoremseparator{.}
	% \theoremprework{\bigskip\hrule}
	% \theorempostwork{\hrule\bigskip}
	\newtheorem{Theorem}{定理}
}
{
	\theoremclass{Theorem}
	\theoremstyle{plain}
	% \theoremheaderfont{\normalfont\bfseries}
	% \theorembodyfont{\slshape}
	% \theoremsymbol{\ensuremath{\diamondsuit}}
	% \theoremseparator{.}
	% \theoremprework{\bigskip\hrule}
	% \theorempostwork{\hrule\bigskip}
	\newframedtheorem{fTheorem}{Theorem}
}
{
	\theoremstyle{plain}
	\theorembodyfont{\slshape}
	\theoremsymbol{\ensuremath{\diamondsuit}}
	\theoremseparator{.}
	% \theoremprework{\bigskip\hrule}
	% \theorempostwork{\hrule\bigskip}
	\newtheorem{Property}{命题}
}
{
	\theoremstyle{plain}
	\theorembodyfont{\slshape}
	\theoremsymbol{\ensuremath{\diamondsuit}}
	\theoremseparator{.}
	\theoremprework{\bigskip\hrule}
	\theorempostwork{\hrule\bigskip}
	\newtheorem{lProperty}{Property}
}
{
	\theoremclass{Property}
	\theoremstyle{plain}
	\newframedtheorem{fProperty}{Property}
}
{
	\theoremstyle{plain}
	% \theoremsymbol{\ensuremath{\heartsuit}}
	\theoremindent0.5cm
	% \theoremnumbering{greek}
	\newtheorem{Lemma}{引理}
}
{
	\theoremindent0cm
	\theoremsymbol{\ensuremath{\spadesuit}}
	\theoremnumbering{arabic}
	\newtheorem{Corollary}[Theorem]{Corollary}
}
{
	\theoremstyle{change}
	\theorembodyfont{\upshape}
	\theoremsymbol{\ensuremath{\ast}}
	\theoremseparator{}
	\newtheorem{Example}{Example}
}
{
	\theoremheaderfont{\upshape\bfseries}
	\theorembodyfont{\mdseries\itshape}
	\theoremstyle{nonumberplain}
	\theoremseparator{}
	\theoremsymbol{\mbox{$\square$}}
	\newtheorem{Proof}{证明}
}
{
	\theoremindent0cm
	\theoremsymbol{\ensuremath{\spadesuit}}
	\theoremnumbering{arabic}
	\theorembodyfont{\upshape}
	\newtheorem{Remark}{注}
}

\newcommand{\myvec}[1]{\vb*{#1}} %定制矢量格式
\newcommand{\myvu}[1]{\vu*{#1}}%单位矢量
%\providecommand{\dd}[0]{\mathrm{d}}
\newcommand{\qst}{\qq{s.t.}}
\newcommand{\qsts}{\qq*{s.t.}}
\newcommand{\tst}{\text{s.t.}}

\newcommand{\RomanNumeralCaps}[1]{\MakeUppercase{\romannumeral #1}}
\newcommand{\e}[1]{\mathrm{e}^{#1}}
\newcommand{\ii}{\mathrm{i}}
\newcommand{\nG}{\ensuremath{\mathrm{G}}}
\newcommand{\gkappa}{\upkappa}

\newcommand{\dvt}[2][]{\dv[#1]{#2}{t}} 
\newcommand{\pdvt}[2][]{\pdv[#1]{#2}{t}} 
\newcommand{\pd}[1]{\pdv*{}{#1}}
\newcommand{\liej}[1]{\ensuremath{{#1}_1,\cdots,{#1}_n}}
\newcommand{\qqiff}{\qq{iff}}
\newcommand{\F}{\ensuremath{\mathscr{F}}}
\newcommand{\La}{\ensuremath{\mathscr{L}}}
\newcommand{\Lad}{\ensuremath{\tilde{\mathscr{L}}}}
\newcommand{\Ha}{\ensuremath{\mathscr{H}}}
\newcommand{\Had}{\ensuremath{\tilde{\mathscr{H}}}}
\newcommand{\Ld}[1]{\ensuremath{\mathcal{L}\indices{_{#1}}}}
\newcommand{\spaceLd}[1]{\ensuremath{\tilde{\mathcal{L}}\indices{_{#1}}}}
\newcommand{\TT}{\ensuremath{\mathscr{T}}}
\newcommand{\I}{\ensuremath{\mathrm{I}}}
\newcommand{\II}{\ensuremath{I}}
\newcommand{\J}{\ensuremath{\mathrm{J}}}
% \newcommand{\hj}[2][-1bp]{\raisebox{#1}{\includegraphics[height=#2 bp]{hj}}}
% \newcommand{\wl}[2][-1bp]{\raisebox{#1}{\includegraphics[height=#2 bp]{wulian}}}

\newcommand{\Nabla}[1]{\tensor{\nabla}{_{#1}}}%协变导数算符
\newcommand{\tNabla}[1]{\tensor{{\tilde{\nabla}}}{_{#1}}}%另一个协变导数算符
\newcommand{\spaceD}[1]{\tensor{D}{_{#1}}}
\newcommand{\Dc}[1]{\tensor{D}{_{#1}}}
\newcommand{\AD}[1]{\tensor{\mathcal{D}}{_{#1}}}
\newcommand{\ADd}{\mathcal{D}}
\newcommand{\Partial}[1]{\tensor{\partial}{_{#1}}}%普通导数算符
\newcommand{\Fd}[2][\tau]{\ensuremath{\frac{\mathrm{D_F}#2}{\dd{#1}}}}%费米导数
\newcommand{\Fdd}[2][\tau]{\ensuremath{\mathrm{D_F}#2/\dd{#1}}}%行内费米导数
\newcommand{\Dd}[2][\tau]{\ensuremath{\frac{\mathrm{D}#2}{\dd{#1}}}}%协变导数
\newcommand{\Ddd}[2][\tau]{\ensuremath{\mathrm{D}#2/\dd{#1}}}%行内协变导数
\newcommand{\ChristoffelSymbol}[3]{\tensor{\Gamma}{^{#1}_{#2}_{#3}}}
\newcommand{\christoffelSymbol}[4]{\frac{1}{2} \tensor{g}{^{#1}^{#4}} \left( \tensor{g}{_{#4}_{#2}_{,#3}} + \tensor{g}{_{#3}_{#4}_{,#2}} - \tensor{g}{_{#2}_{#3}_{,#4}} \right) }
\newcommand{\riemannR}[5]{\tensor{\Gamma}{^{#4}_{#1}_{#3}_{,#2}}- \tensor{\Gamma}{^{#4}_{#2}_{#3}_{,#1}}+ \ChristoffelSymbol{#5}{#3}{#1} \ChristoffelSymbol{#4}{#2}{#5}- \ChristoffelSymbol{#5}{#3}{#2} \ChristoffelSymbol{#4}{#1}{#5}}
%\newcolumntype{Y}{>{\centering\arraybackslash}X}
%\newcommand{\tmu}{\mspace{2mu}}
%\newcommand{\mmacs}{\bfseries\ttfamily\itshape\color[RGB]{67,137,88}}%mma局部变量字体
%\newcommand{\mmab}{\bfseries\ttfamily\color[RGB]{60,125,145}}%mma变量字体
%\newcommand{\mmaundef}{\bfseries\ttfamily\color[RGB]{0,44,195}}%mma未定义
%\newcommand{\mma}{\bfseries\ttfamily}%mma代码字体

% \newcommand{\Rn}[1]{\ensuremath{\mathbb{R}^{#1}}}
\newcommand{\SO}[1]{\ensuremath{\mathrm{SO}\left(#1\right)}}
\newcommand{\SU}[1]{\ensuremath{\mathrm{SU}\left(#1\right)}}
\newcommand{\OO}[1]{\ensuremath{\mathrm{O}\left(#1\right)}}
\newcommand{\so}[1]{\ensuremath{\mathrm{so}\left( #1 \right)}}
\newcommand{\su}[1]{\ensuremath{\mathrm{su}\left( #1 \right)}}
\newcommand{\SL}[1]{\mathrm{SL}\left( #1 \right)}
\newcommand{\sll}[1]{\mathrm{sl}\left( #1 \right)}
\newcommand{\Ad}[1]{\mathrm{Ad}_{#1}}
\newcommand{\Hom}[2]{\mathrm{Hom}\left( #1,#2 \right)}
\newcommand{\Cyl}{\mathrm{Cyl}}

\newcommand{\Lo}[1]{\ensuremath{\mathrm{Lor}\left(#1\right)}}
\newcommand{\Riem}[1]{\ensuremath{\mathrm{Riem}\left( #1 \right)}}
\newcommand{\superspace}[1]{\ensuremath{\mathcal{S}\left(#1\right)}}
\newcommand{\Diff}[1]{\ensuremath{\mathrm{Diff}\left(#1\right)}}
\newcommand{\definedby}{:=}

\newcommand{\phasespace}[1]{\mathcal{#1}}
\newcommand{\configurationspace}[1]{\mathcal{#1}}
\newcommand{\extentedconfigurationspace}[1]{\bar{\configurationspace{#1}}}
\newcommand{\staralgebra}[1]{\mathfrak{#1}}
\newcommand{\Linear}[1]{\mathcal{L}\left( {#1} \right)}
\newcommand{\pathorder}{\hat{\mathcal{P}}}
\newcommand{\Dkin}{\mathcal{D}_{\text{kin}}}
\newcommand{\DDiff}{\mathcal{D}_{\text{Diff}}}
\newcommand{\Dphys}{\mathcal{D}_{\text{phys}}}
\newcommand{\Hil}{\mathcal{H}}
\newcommand{\Hkin}{\Hil_{\text{kin}}}
\newcommand{\HDiff}{\Hil_{\text{Diff}}}
\newcommand{\Hphys}{\Hil_{\text{phys}}}
\newcommand{\masterconstraint}[1]{\ensuremath{\bm{\mathsf{#1}}}}
\newcommand{\complex}[1]{\mathcal{#1}}
\newcommand{\amplitude}{A}

\newcommand{\form}[1]{\ensuremath{\bm{#1}}}
\newcommand{\Ric}{\ensuremath{R}}
\newcommand{\curR}{\ensuremath{{R}}}
\newcommand{\spacecurR}{\ensuremath{\mathcal{R}}}
\newcommand{\vol}{\ensuremath{\varepsilon}}
\newcommand{\nvol}{\ensuremath{\epsilon}}
\newcommand{\spc}{\ensuremath{\mathcal{S}}}
\newcommand{\setofcurve}{\mathcal{C}}
\newcommand{\setofpath}{\mathcal{P}}

\newcommand{\TB}[2][{}]{\mathrm{T}_{#1}\!{#2}}
\newcommand{\TBx}[2][{}]{\mathrm{T}_{#1} {#2}}
\newcommand{\CTB}[2][{}]{\mathrm{T}^*_{#1}\!{#2}}
\newcommand{\FB}[2][{}]{\mathrm{F}_{#1}\!{#2}}
\newcommand{\Jet}[4][{}]{\ensuremath{\mathrm{Jet}_{#1}^{#2}\left( #3,#4 \right)}}

\newcommand{\End}[1]{\mathrm{End}\left( #1 \right)}

% \newcommand{\uphbar}{\mathchar'0026\mkern-9mu h}
\newcommand{\uphbar}{\hbar}

%\renewcommand{\CancelColor}{\color{blue!70!red}}

%\newcommand*{\circled}[1]{\lower.7ex\hbox{\tikz\draw (0pt, 0pt)circle (.5em) node {\makebox[1em][c]{\small #1}};}}
\newcommand{\card}[1]{\abs{#1}}

\DeclareMathOperator{\ddiv}{div}
\DeclareMathOperator{\ggrad}{grad}
\DeclareMathOperator{\ccurl}{curl}
\DeclareMathOperator{\Exp}{Exp}
\DeclareMathOperator{\Span}{Span}
\DeclareMathOperator{\Inv}{Inv}
\DeclareMathOperator{\image}{im}

\makeatletter
\DeclareFontFamily{U}  {MnSymbolF}{}
\DeclareSymbolFont{symbolsMN}{U}{MnSymbolF}{m}{n}
\SetSymbolFont{symbolsMN}{bold}{U}{MnSymbolF}{b}{n}
\DeclareFontShape{U}{MnSymbolF}{m}{n}{
    <-6>  MnSymbolF5
   <6-7>  MnSymbolF6
   <7-8>  MnSymbolF7
   <8-9>  MnSymbolF8
   <9-10> MnSymbolF9
  <10-12> MnSymbolF10
  <12->   MnSymbolF12}{}
\DeclareFontShape{U}{MnSymbolF}{b}{n}{
    <-6>  MnSymbolF-Bold5
   <6-7>  MnSymbolF-Bold6
   <7-8>  MnSymbolF-Bold7
   <8-9>  MnSymbolF-Bold8
   <9-10> MnSymbolF-Bold9
  <10-12> MnSymbolF-Bold10
  <12->   MnSymbolF-Bold12}{}
\DeclareMathSymbol{\tbigtimes}{\mathop}{symbolsMN}{2}
\newcommand*{\bigtimes}{%
  \DOTSB
  \tbigtimes
  \slimits@ 
}
\makeatother

\makenomenclature
\renewcommand{\nomname}{符号说明}
